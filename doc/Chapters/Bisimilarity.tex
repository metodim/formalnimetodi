\section{Bisimulation minimization and equivalence}
% no \IEEEPARstart
In theoretical computer science a bisimulation is a binary relation between labeled transition systems, 
associating systems which behave in the same way in the sense that one system simulates the other and
vice-versa. The bisimulation equivalence finds its extensive application in the area of formal
verification of concurrent systems, for example to check the equivalence of an implementation of a
certain system with respect to its specification model. 

The process of finding a bisimulation equivalence between two labeled transition systems includes 
two main steps:
\begin{itemize}
\item minimizing each of the two LTSs to its canonical form
\item performing a comparison between the two canonical forms obtained in the first step
\end{itemize}

This chapter is concerned with both of these steps. The reduction of a labeled transition system
to a canonical form was implemented with two different methods: the so called naive method, and
advanced method according to Fernandez [Fer89].