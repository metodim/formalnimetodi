\section{Preliminaries}
\label{sec:preliminaries}

In this section, we give some formal definitions of the basic terminology used throughout the paper \cite{ReactiveSystems}\cite{ModellingAndAnalysis}\cite{WeakBisimulation}.

\begin{definition}
(Calculus of Communicating Systems): Calculus of Communicating Systems is algebraic theory to formalize the notion of concurrent computation. Commonly known as CCS.
\end{definition}

\begin{definition}
(Labeled transition system): A labeled transition system is a 5-tuple $A=\left(S, Act, \rightarrow, s_{0}, F \right)$ where
\begin{itemize}
	\item $S$ is set of $states$,
	\item $Act$ is a set of actions, possibly multi-actions,
	\item $\rightarrow \subseteq S \times Act \times S$ is a transition relation,
	\item $s_{0} \in S$ is the initial state,
	\item $F \subseteq S$ is the set of terminating states.	
\end{itemize}
\end{definition}

\begin{definition}
(Strong bisimulation equivalence): Let $A_{1}=\left(S_{1}, Act, \rightarrow_{1}, s_{1}, F_{1}\right)$ and $A_{2}=\left(S_{2}, Act, \rightarrow_{2}, s_{2}, F_{2}\right)$ be labeled transition systems. A binary relation $\mathcal{R} \subseteq S_{1} \times S_{2}$ is called a strong bisimulation relation iff for all $s \in S_{1}$ and $t \in S_{2}$ such that $s\mathcal{R} t$ holds, it also holds for all actions $a \in Act$ that:
\begin{enumerate}
\item if $s\stackrel{a}{\rightarrow}_{1}s'$ then there is a transition $t' \in S_{2}$ such that $t\stackrel{a}{\rightarrow}_{2}t'$ with $s'\mathcal{R} t'$,
\item if $t\stackrel{a}{\rightarrow}_{2}t'$ then there is a transition $s' \in S_{1}$ such that $s\stackrel{a}{\rightarrow}_{1}s'$ with $s'\mathcal{R} t'$, and
\item $s \in T_{1}$ iff $t \in T_{2}$.
\end{enumerate}
Two states $s$ and $s'$ are strongly bisimilar, written $s\sim s'$, if there is a strong bisimulation equivalence $\mathcal{R}$ that relates them.
\end{definition}

\begin{definition}
Let $P$ and $Q$ be two processes or, more generally, states in a labeled transition system. For each action $a$,
$P\stackrel{a}{\Rightarrow}Q$ is a weak transition iff:
\begin{itemize}
	\item either $a\neq\tau$ and there are processes $P'$ and $Q'$ such that $P\left(\stackrel{\tau}{\rightarrow}\right)^{*}P'\ \stackrel{a}{\rightarrow}\ Q'\left(\stackrel{\tau}{\rightarrow}\right)^{*}Q$,
	\item or $a=\tau$ and $P\left(\stackrel{\tau}{\rightarrow}\right)^{*}Q$,
\end{itemize}
where $\left(\stackrel{\tau}{\rightarrow}\right)^{*}$ denotes the reflexive and transitive closure of the relation $\stackrel{\tau}{\rightarrow}$.
\end{definition}

\begin{definition}
(Weak bisimulation equivalence): A binary relation $\mathcal{R}$ over the set of states of a labeled transition system is a weak bisimulation (observational equivalence) iff, whenever $s_{1}\mathcal{R} s_{2}$ and $a$ is an action (including $\tau$):
\begin{enumerate}
	\item if $s_{1}\stackrel{a}{\rightarrow}s_{1}'$ then there is a weak transition $s_{2}\stackrel{a}{\Rightarrow}s_{2}'$ such that $s_{1}'\mathcal{R} s_{2}'$;
	\item if $s_{2}\stackrel{a}{\rightarrow}s_{2}'$ then there is a weak transition $s_{1}\stackrel{a}{\Rightarrow}s_{1}'$ such that $s_{1}'\mathcal{R} s_{2}'$;
\end{enumerate}
Two states $s$ and $s'$ are observationally equivalent (or weakly bisimilar), written $s\approx s'$, iff there is a weak bisimulation equivalence $\mathcal{R}$ that relates them.
\end{definition}

\begin{definition}
(Saturation): Let $A=\left(S, Act, \rightarrow, s, F \right)$ be a labeled transition system. Then a saturation of $A$ is
\begin{align*}
 A^{*} &=\left\{\left(p,a,q\right)| p\stackrel{a}{\Rightarrow}q\right\}= \\
 &=A\cup\left(\stackrel{a}{\rightarrow}\right)^{*}\cup\left\{\left(p,a,q\right)| a\neq\tau\wedge\left(\exists p',q'\in Act\right) p\left(\stackrel{\tau}{\rightarrow}\right)^{*}p'\stackrel{a}{\rightarrow}q'\left(\stackrel{\tau}{\rightarrow}\right)^{*}q\right\}.
\end{align*}
Two labeled transition systems are weakly bisimilar iff their saturated labeled transition systems are strongly bisimilar. 
\end{definition}

\begin{definition}
(Hennessy-Milner logic): The set of Hennessy-Milner formulas over a set of actions $Act$ is given by the following abstract syntax:
\begin{equation*}
  F,G::=\mathit{tt} | \mathit{ff} | F\wedge G | F \vee G|\left\langle a \right\rangle F|\left[ a \right] F
\end{equation*}
where $a\in Act$ and $\mathit{tt}$ and $\mathit{ff}$ denote $true$ and $false$, respectively. 
If $A=\left\{ a_{1},...,a_{n} \right\} \subseteq Act \left( n \geq 0\right)$, 
we use the abbreviation $\left\langle A \right\rangle F$ for the formula 
$\left\langle a_{1} \right\rangle F \vee ... \vee \left\langle a_{n} \right\rangle F$ and 
$\left[ a \right] F$ for the formula 
$\left[ a_{1} \right] F \wedge ... \wedge \left[ a_{n} \right]F$. 
(If $A=\emptyset$ then $\left\langle a\right\rangle F=\mathit{ff}$ and $\left[a\right] = \mathit{tt}$.)
\end{definition}
\vspace{3mm}