\section{$U^w$ and $U^s$ implementation}

This part explains how implementation of Hennessy-Milner logic works. For a given Hennessy-Milner expression and a graph, we should get an output whether that expression is valid for that graph.  One Hennessy-Milner expression can be made of these set of tokens \{"AND", "OR", "UW", "US", "NOT", "[", "]", "<", ">", "TT", "FF", "(", ")", "{", "}", ","\}.  The first part of this process is called tokenization. Tokenization is the process of demarcating and possibly classifying sections of a string of input characters. The resulting tokens are then passed on to some other form of processing.
The process can be considered as a sub-task of parsing input. As the tokens are being read they are proceeded to the parser. The parser is a LR top-down parser. This parser is able to recognize a non-deterministic grammar which is essential for the evaluation of the Hennessy-Milner expression. As the parser is reading the tokens, thus in a way we “move” through the graph and determine if some of the following steps are possible. Every condition is in a way kept on stack and that enables later return to any of the previous conditions. The process is finished when the expression  is processed or when it is in a condition from which none of the following actions can be taken over.  
