% This is LLNCS.DEM the demonstration file of
% the LaTeX macro package from Springer-Verlag
% for Lecture Notes in Computer Science,
% version 2.4 for LaTeX2e as of 16. April 2010
%

\documentclass{llncs}
%
\usepackage{makeidx}  % allows for indexgeneration
\usepackage[pdftex]{graphicx}
% declare the path(s) where your graphic files are
\graphicspath{{./Images/}}
% and their extensions so you won't have to specify these with
% every instance of \includegraphics[scale=•]{•}
\DeclareGraphicsExtensions{.jpeg,.jpg,.png}

\usepackage{url}
\urldef{\mailsa}\path|jane.jovanovski@gmail.com, maja.siljanoska@gmail.com, carevski@gmail.com, dragan.sahpaski@gmail.com, pgorcevski@t-home.mk, metodi.micev@gmail.com, b.ilijoski@gmail.com, vlatko.gmk@gmail.com|
\usepackage[pdfpagelabels,hypertexnames=false,breaklinks=true,bookmarksopen=true,bookmarksopenlevel=2]{hyperref}
\usepackage{algorithm}
\usepackage{algorithmic}
\usepackage{amsmath}
\usepackage{subfigure}

\begin{document}

\mainmatter

%
% paper title
% can use linebreaks \\ within to get better formatting as desired
\pagestyle{plain}
\thispagestyle{plain}
\pagenumbering{arabic}

\title{A tool for modeling, manipulation and analysis of concurrent systems based on CCS process language and LTS as its semantic model}
\subtitle{(Technical Report)\footnote{This technical report is a result of a group project work for the course in Formal methods at the MSc studies in Computer Sciences at the Institute of Informatics 2010/2011}}

% author names and affiliations

\author{
J. Jovanovski \and M. Siljanoska\and V. Carevski 
\and D. Sahpaski \and P. Gjorcevski \and \\M. Micev 
\and B. Ilijoski \and V. Georgiev  
}  

% the affiliations are given next
\institute{Institute of Informatics,\\
Faculty of Natural Sciences and Mathematics,\\
University "Ss. Cyril and Methodius",\\
Skopje, Macedonia\\
\mailsa
}

\maketitle

\begin{abstract}
This paper reports on an effort to build a tool for modeling, specification, and analysis of concurrent systems. The tool implements the CCS process language and can build labeled transition systems from CCS expressions. Furthermore, it can be used to reduce the state space of a labeled transition system and to check whether two labeled transition systems exhibit the same behaviour, using strong and weak bisimulation equivalence. It can parse Henessy-Milner logic formulas and also implements the 'strong until' and 'weak until' operators. The tool has the functionality needed to perform modeling, specification, and verification, illustrated on two classical examples in the concurrency theory: the Alternating Bit Protocol and Peterson's mutual exclusion algorithm.

\keywords{CCS, labeled transition system, bisimulation equivalence, minimization, comparison, formal verification, Henessy-Milner logic.}

\end{abstract} % Abstract

\section{Introduction}

Due to the increasing complexity of the real-life systems, manual analysis is very hard and often impossible. Therefore, having computer support for modeling, analysing and verifying system behaviour is of an essential importance nowdays. This means that the systems properties and their behaviour need to be described by formal methods with an explicit formal semantics.

One of the basic process languages used to formally describe and analyse any collection of interacting processes is the Calculus of Communication Systems (CCS), process algebra introduced by Robert Milner and based on the message-passing paradigm. As a semantic model of CCS the notion of Labelled Transition System (LTS) is used. CCS, like any process algebra, can be exploited to describe both implementations of processes and specifications of their expected behaviours. 

One of the most important behavioral equivalence is the strong bisimulation equivalence (strong bisimilarity) because if two processes and bisimilar, they cannot be distinguished by any realistic form of behavioral observation. Another useful behavioral equivalence is the weak bisimulation equivalence (weak bisimilarity) which relates processes with internal (non-observable) actions. These and other behavioral equivalences can be used to reduce the size of the state space of an LTS and to check the equivalence between two LTSs. This finds an extensive application in model checking and formal verification.

This paper presents a simple tool that can be used to automate the process of modeling, manipulation and analysis of concurrent systems described by the CCS process language. The tool is given as a Java executable jar library with very simple Graphical User Interface (GUI). Even though it is simple, the tool is functional enough to be able to satisfy the basic requirements for modeling, specification and verification. It can parse CCS expressions, generate LTS from CCS in Aldebaran format, reduce an LTS to its canonical form modulo strong or weak bisimilarity and check whether two LTSs are strongly or weakly bisimilar. Additionally, it can parse HML formulas and implements the HML operators for recursive definitions as well.

\subsection{Related work} 

Different tools for modeling, specification and verification of concurrent reactive systems have been developed over the past two decades. Probably the most famous and most commonly used ones, especially in the academic environment, are the Edinburgh Concurrency Workbench (CWB) and micro Common Representation Language 2 (mCRL2). 

The Edinburgh Concurrency Workbench (CWB) is a tool for analysis of cuncurrent systems. CWB allows for equivalence, preorder and model checking using a variety of different process semantics. It also allows defining of behaviours in an extended version of CCS and perform various analysis of these behaviours, such as checking various semantic equivalences for examples checking if two processes (agents in CWB) are strongly or weakly bisimilar \cite{CWB}. We have used the CWB many times while testing our tool, for checking CCS validity and also testing algorithms for strong and weak bisimularity of LTS graphs. Although CWB covers much of the functionality of our tool and more, CWB has a command interpreter interface that is more difficult to work with, unlike our tool that has a Graphical User Interface (GUI), and is very intuitive. As far as we know the CWB does not have the functionality for exporting LTS graphs in Aldebaran format, that our tool has. 

The micro Common Representation Language 2 (mCRL2), successor of CRL, is a formal specification language that can be used to specify and analyse the behaviour of distrubuted systems and protocols. Its accompanying toolset contains extensive collection of tools to automatically translate any mCRL2 specification to a linear process, manipulate and simulate linear processes, generate the state space associated with a linear process, manipulate and visualize state spaces, generate a PBES from a formula and a linear process, generate a BES from PBES and manipulate and solve (P)BESs \cite{mCRL2}. All mCRL2 tools can be used from the command line, but mCRL2 has an enhanced Graphical User Interface (GUI) as well which makes it very user-friendly and easy to use. We have used mCRL2 many times while testing our tool, for testing the algorithms for minimization and comparison of LTS graphs using strong and weak bisimularity. However, to our knowledge, mCRL2 does not provide the possibility to define systems' behaviour in the CCS process language nor to specify systems' properties using HML logic, functionalities that are implemented in the very initial version of our tool. Also, even though mCRL2 supports minimization modulo strong and weak bisimulation equivalence, it does not output the computed bisimulation, feature that we have implemented in our tool.  
 
\subsection{Outline} The contributions of this paper are organized as follows. In the next section we give introduction to some of the basic terminology used throughout the report. The third section is devoted to the implementation of the CCS process language and generation of Labelled Transition Systems as a semantic model of process expressions, and it also discusses some of the choices made during the implementation. Section four describes the process of reducing the size of the state space of an LTS as well as checking the equivalence between two LTSs with respect to behavioural relations such as strong bisimilarity and observational equivalence (weak bisimilarity). It includes implementation details of two algorithms for computing strong bisimulation equivalence, the naive algorithm and the advanced algorithm due to Fernandez, as well as description of the saturation technique together with a respective algorithm  for saturating an LTS with which the problem of computing weak bisimilarity is reduced to computing strong bisimilarity over the saturated systems. Section five explains how the implementation of Hennesy-Milner Logic (HML) works and gives implementation details for the $U^{w}$ and $U^{s}$ operators. Next we illustrate the application of the tool for modeling, specification and verification of some classical examples. These classical examples include the Alternating Bit Protocol and Peterson's mutual exclusion algorithm. Finally, we give some conclusions and some directions for future development of the tool. We also include four appendixes. The first appendix contains the experimental results we got with analysis of the running times of our implementations of the two algorithms for computing bisimulation equivalence. Appendix two shows the syntax diagram for the grammar that recognizes CCS expressions, while Appendix three presents the grammar for HML expressions. The last appendix includes some visual illustration of the tool's usage via screenshots of the graphical application.
 % Chapter1
\section{Preliminaries}

In this section, we give some preliminaries of the basic terminology used throughout the report.

\begin{definition}
(Calculus of Communicating Systems): Calculus of Communicating Systems is algebraic theory to formalize the notion of concurrent computation. Commonly known as CCS.
\end{definition}

\begin{definition}
(Labelled Transition System): A labelled transition system (LTS) is a five tuple $A=\left(S, Act, \rightarrow, s, T \right)$ where
\begin{itemize}
	\item $S$ is set of $states$,
	\item $Act$ is a set of actions, possibly multi-actions,
	\item $\rightarrow \subseteq S \times Act \times S$ is a $transition relation$,
	\item $s \in S$ is the $initial state$,
	\item $T \subseteq S$ is the set of $terminating states$.	
\end{itemize}
\end{definition}

\begin{definition}
(Strong bisimulation): Let $A_{1}=\left(S_{1}, Act, \rightarrow_{1}, s_{1}, T_{1}\right)$ and $A_{2}=\left(S_{2}, Act, \rightarrow_{2}, s_{2}, T_{2}\right)$ be labelled transition systems. A binary relation $R \subseteq S_{1} \times S_{2}$ is called a strong bisimulation relation iff for all $s \in S_{1}$ and $t \in S_{2}$ such that $sRt$ holds, it also holds for all actions $a \in Act$ that:
\begin{enumerate}
\item if $s\stackrel{a}{\rightarrow}_{1}s'$ then there is a $t' \in S_{2}$ such that $t\stackrel{a}{\rightarrow}_{2}t'$ with $s'Rt'$,
\item if $t\stackrel{a}{\rightarrow}_{2}t'$ then there is a $s' \in S_{1}$ such that $s\stackrel{a}{\rightarrow}_{1}s'$ with $s'Rt'$, and
\item $s \in T_{1}$ if and only in $t \in T_{2}$.
\end{enumerate}
\end{definition}

\begin{definition}
Let $P$ and $Q$ be CCS processes or, more generally, states in a labeled transition system. For each action $a$, we shall write
$P\stackrel{a}{\Rightarrow}Q$ iff:
\begin{itemize}
	\item Either $a\neq\tau$ and there are processes $P'$ and $Q'$ such that $P\left(\stackrel{\tau}{\rightarrow}\right)^{*}P'\ \stackrel{a}{\rightarrow}\ Q'\left(\stackrel{\tau}{\rightarrow}\right)^{*}Q$,
	\item Or $a=\tau$ and $P\left(\stackrel{\tau}{\rightarrow}\right)^{*}Q$,
\end{itemize}
where we write $\left(\stackrel{\tau}{\rightarrow}\right)^{*}$ for he reflexive and transitive closure of the relation $\stackrel{\tau}{\rightarrow}$.
\end{definition}

\begin{definition}
(weak bisimulation and observational equivalence): A binary relation $R$ over the set of states of an LTS is a weak bisimulation iff, whenever $s_{1}Rs_{2}$ and $a$ is an action (including $\tau$):
\begin{itemize}
	\item If $s_{1}\stackrel{a}{\rightarrow}s_{1}'$ then there is a transition $s_{2}\stackrel{a}{\Rightarrow}s_{2}'$ such that $s_{1}'Rs_{2}'$;
	\item If $s_{2}\stackrel{a}{\rightarrow}s_{2}'$ then there is a transition $s_{1}\stackrel{a}{\Rightarrow}s_{1}'$ such that $s_{1}'Rs_{2}'$;
\end{itemize}
\end{definition}

\begin{definition}
(Hennessy Milner formulae): The set $M$ Hennessy Milner formulae over a set of actions $Act$ is given by the following abstract syntax:
\begin{equation*}
  F,G::=tt |ff | F\wedge G | F \vee G|\left\langle a \right\rangle F|\left[ a \right] F
\end{equation*}
where $a \in Act$ and we use $tt$ and $ff$ to denoto true and false, respectively. 
If $A=\left\{ a_{1},...,a_{n} \right\} \subseteq Act \left( n \geq 0\right)$, 
we use use the abbreviation $\left\langle A \right\rangle F$ for the formula 
$\left\langle a_{1} \right\rangle F \vee ... \vee \left\langle a_{n} \right\rangle F$ and 
$\left[ a \right] F$ for the formula 
$\left[ a_{1} \right] F \wedge ... \wedge \left[ a_{n} \right]F$. 
(If $A=\emptyset$ then $\left\langle a\right\rangle F=ff$ and $\left[a\right] = tt$.)
\end{definition} % Chapter2
\section{CCS parsing and labeled transition system generation}
\label{sec:parsing}

Given a set of action names, the set of CCS processes is defined by the following BNF grammar:

\begin{equation}\label{eq:ccs_bnf}
P ::= \emptyset \hspace{1 mm} | \hspace{1 mm} a.P_{1} \hspace{1 mm} | \hspace{1 mm} A \hspace{1 mm} | \hspace{1 mm}P_{1}+P_{2} \hspace{1 mm} |
\hspace{1 mm} P_{1} | P_{2} \hspace{1 mm} | \hspace{1 mm} P_{1}[b/a] \hspace{1 mm} | \hspace{1 mm} P_{1} \backslash a
\end{equation}

Two choices were considered for the problem of building a parser in TMACS. The first choice was to build
a new parser, which required much more resources and the second choice was to define a grammar
and to use a parser generator (compiler-compiler, compiler generator) software to generate the parser source code.
In formal language theory, a context-free grammar \cite{Chomsky} is a grammar in which every production 
rule has the form 
$V \rightarrow w$,
where V e is a nonterminal symbol, and w is a string of terminal and/or nonterminal symbols 
(w can be empty). Obviously, the defined BNF grammar for describing the CCS process is a CFG. 
Deterministic context-free grammars \cite{Chomsky} are grammars that can be recognized by a 
deterministic pushdown automaton or equivalently, grammars that can be recognized by a LR (left to right) 
parser \cite{Compilers}. Deterministic context-free grammars are a proper subset of the context-free grammar 
family \cite{Chomsky}. %Although, the defined grammar is a non-deterministic context-free grammar, modern parsers have a look ahead feature, which means that the parser will not make a parsing decision until it reads ahead several tokens. This feature alows us to use a parser generator software, that will generate LR or LL parser which is able to recognize the non-deterministic grammar.

\subsection{Generating the parser}
ANTLR (ANother Tool for Language Recognition) \cite{ANTLR} is a parser generator that was used 
to generate the parser for the CCS grammar in TMACS. ANTLR uses LL(*) parsing and also allows generating 
parsers, lexers, tree parsers and combined lexer parsers. It also automatically generates 
abstract syntax trees \cite{Compilers}, by providing operators and rewrite rules to guide the tree construction,
which can be further processed with a tree parser, to build an abstract model of the AST using some programming language.
%A language is specified by using a context-free grammar which is expressed using Extended Backus Naur Form (EBNF) \cite{Compilers}. In short, an LL parser is a top-down parser which parses the input from left to right and constructs a leftmost derivation of the sentence. An LL parser is called an LL(*) parser if it is not restricted to a finite k tokens of lookahead, but can make parsing decisions by recognizing whether the following tokens belong to a regular language. This is achieved by building temporary deterministic finite automaton, which are maintained and used until the parser makes a decision. This feature and some more optimization for the LL parser were published in recent years which made this kind of parser generators popular and favorable \cite{Compilers}.

ANTLR was used to generate lexer, parser and a well defined abstract syntax tree which represents a tree representation 
of the abstract syntactic structure of the parsed sentence. The term abstract is used in a sense 
that the tree does not represent every single detail that exists in the language syntax, 
e.g., parentheses are implicit in the tree structure. From a development point of view it is 
much easier to work with trees than with a list of recognized tokens. One example of an abstract syntax tree
is shown in Fig.~\ref{fig:ast_example} which is the result of parsing the expression: 

\begin{equation}\label{eq:ccs_example}
 A=( b.B \hspace{1 mm} | \hspace{1 mm} c.D \hspace{1 mm} + \hspace{1 mm} d.D )
\backslash \left\lbrace b, \hspace{1 mm} c \right\rbrace 
\end{equation}

\begin{figure}[h]
\centering
\includegraphics[width=3.0in]{ast_example}
\caption{Example of an abstract syntax tree}
\label{fig:ast_example}
\end{figure}

\subsection{CCS domain model and labeled graph generation}
Although working directly with abstract syntax trees and performing all algorithms on them is possible, it causes 
a limitation in future changes, where even a small change in the grammar and/or in the structure of the
generated abstract syntax trees causes a change in the implemented algorithms. Because of this,
a specific domain model was built along with a domain builder algorithm, which has corresponding
abstractions for all CCS operators, processes and actions. The input of the domain builder algorithm 
is an abstract syntax tree, and the output is a fully built domain model. 
%The algorithms for constructing a labeled graph were implemented on the domain model because it is not expected for the domain model structure to change much in the future. The domain model is also a tree-like structure, so it is as easy to work with, as with the abstract syntax tree. 

The algorithm for generation of labeled transition system implemented in TMACS is a recursive algorithm which 
traverses the tree structure of objects in the domain model and performs SOS rule every time it reaches an operation.
In this fashion all SOS transformation are performed on the domain and as result a new graph
structure is created which represents the labeled transition system which can be easily exported to a file
in Aldebaran format.

%\subsection{Workflow of operations}
%In Fig.~\ref{fig:workflow} the workflow of all operations that are executed for constructing a labeled transition system graph from a 
%CCS expression is shown. Every operation is done as a standalone algorithm independent from the other operations,
%that has input and output shown in the figure. The modular design was deliberately chosen in order to help achieve better 
%testing and maintenance of the source code. 


%\begin{figure}[h]
%\centering
%\includegraphics[height=3.0in]{workflow}
%\caption{Workflow of all operations for producing a labeled transition system graph from a CCS expression}
%\label{fig:workflow}
%\end{figure}
 % Chapter3
\section{Minimization and comparison of labeled transition systems}
\label{sec:bisimulation}

Bisimulation equivalence (bisimilarity) \cite{Park} is a binary relation between labeled transition systems which associates systems that can simulate each other's behaviour in a stepwise manner, enabling comparison of different transition systems \cite{ModelChecking}. An alternative perspective is to consider the bisimulation equivalence as a relation between states of a single labeled transition system. Using the quotient transition system under such a relation, smaller models of the labeled transition system can be obtained \cite{ModelChecking}.

%The bisimulation equivalence finds its extensive application in many areas of computer science such as concurrency theory, formal verification, set theory, etc. For instance, in formal verification minimization with respect to bisimulation equivalence is used to reduce the size of the state space to be analyzed. Also, bisimulation equivalence is of particular interest in model checking, in specific to check the equivalence of an implementation of a certain system with respect to its specification model.

TMACS implements both options: reducing the size of the state space of a given labeled transition system and checking the equivalence of two labeled transition systems, using two behavioral equivalence relations: strong bisimilarity and observational equivalence (weak bisimilarity).

\subsection{Minimization of a labeled transition system modulo strong bisimilarity}
The process of reducing the size of the state space of a labeled transition system $A=\left(S, Act, \rightarrow, s, F \right)$ in TMACS was implemented using an approach which consists of two steps:
\begin{enumerate}
\item Computing strong bisimulation equivalence (strong bisimilarity) for the labeled transition system;
\item Minimizing the labeled transition system to its canonical form using the strong bisimilarity obtained in the first step;
\end{enumerate}

Two different methods were used for computing the strong bisimulation equivalence: the so called naive method and a more efficient 
method due to Fernandez, both of which afterwards serve as minimization procedures.

The naive algorithm for computing strong bisimulation over finite labeled transition systems stems directly from the theory of partially ordered sets and lattices \cite{ReactiveSystems} underlying Tarski's classic fixed point theorem \cite{Tarsky}. It is based on an alternative definition of the notion of strong bisimilarity as the largest fixed point of the monotic function $\mathcal{F}$ defined as follows \cite{ReactiveSystems}:\\
\\
Let $\mathcal{R}$ is a binary relation over $\mathit{S}$, that is, an element of the set $2^{\mathit{S}\times \mathit{S}}$. Then $(p,q)\in \mathcal{F(R)}$ for all $p,q\in \mathit{S}$ iff:
	\begin{enumerate}
		\item $p\stackrel{a}{\rightarrow}p'$ implies $q\stackrel{a}{\rightarrow}q'$ for some $q'$ such that $(p',q')\in\mathcal{R}$;
		\item $q\stackrel{a}{\rightarrow}q'$ implies $p\stackrel{a}{\rightarrow}p'$ for some $p'$ such that $(p',q')\in\mathcal{R}$;
	\end{enumerate}

The algorithm that stems from this interpretation of the strong bisimulation equivalence has time complexity of $O(mn)$ for a labeled transition system with \emph{m} transitions and \emph{n} states. Its implementation in TMACS takes as an input a labeled transition system in Aldebaran format, and generates the corresponding labeled graph as a list of nodes representing the states which use the following data structure:
\begin{itemize}
	\item $S_p=\{(a, q)\hspace{1mm}|\hspace{1mm} p\stackrel{a}{\rightarrow}q, \hspace{1mm} p,q\in \mathit{S}, \hspace{1mm} a\in \mathit{Act}\}$ - set of pairs $(a, q)$ for state $p$ where $a$ is an outgoing action for $p$ and $q$ is a state
	reachable from $p$ with the action $a$
\end{itemize}
The algorithm then computes the strong bisimulation equivalence and outputs it as pairs of bisimilar states:
\begin{equation*}
	L = \left\{\left(p,q\right)|\hspace{1mm}\hspace{1mm} p\sim q, \hspace{1mm}p,q\in \mathit{S}\right\}
\end{equation*}

The algorithm due to Fernandez exploits the idea of the relationship between strong bisimulation equivalence 
and the relational coarsest partition problem \cite{KanellakisSmolka}\cite{PaigeTarjan}. Paige and Tarjan \cite{PaigeTarjan} 
proposed an algorithm that computes the relational coarsest partition problem in $O(m \log n)$ time and $O(m)$ space for a labeled transition system with $m$ transitions and $n$ states. Fernandez adapted the algorithm of Paige-Tarjan by considering a family of relations $\left(T_a\right)_{a\in \mathit{Act}}$ instead of one relation, with $T_a=\{(p,q)|p\stackrel{a}{\rightarrow}q\}$ as a transition relation for action $a\in \mathit{Act}$ \cite{Fernandez}. The adapted version has the same $O(m \log n)$ time complexity as the original one, major difference being that a refinement step is made with only one element of $Act$ in the original one. 

Our implementation of Fernandez's algorithm in TMACS takes a labeled transition system in Aldebaran format as an input, generates a labeled graph and then partitions the labeled graph into its coarsest blocks where each block represents a set of bisimilar states. Partition is a set of mutually exclusive blocks whose union constitutes the graph universe \cite{Fernandez}. To define graph states 
and transitions we used the following terminology represented by suitable data structures: 
\begin{itemize}
	\item $T_a[p]=\{q\}$ - an $a$-transition from state $p$ to state $q$
	\item $T_a{}^{-1}[q]=\{p\}$ - an inverse $a$-transition from state $q$ to state $p$
	\item $T_a{}^{-1}[B]=\cup \left\{T_a{}^{-1}[q],q\in B\right\}$ - inverse transition for block $B$ and action $a$
	\item $W$ - set of sets called splitters that are being used to split the partition
	\item infoB$(a, p)$ - info map for block $B$, state $p$ and action $a$
\end{itemize}

The algorithm of Fernandez outputs the strong bisimulation equivalence relation over $\mathit{Proc}$ as a partition $P=\left\{B_{1},...,B_{n}\right\}$ where $B_{i}$, $i=\overline{1,n}$, represent its equivalence classes:
\begin{equation*}
	P = \left\{B_{i}\hspace{1mm}|\hspace{1mm} p\approx q,\hspace{1mm} \forall p,q\in B_{i}, i=\overline{1,n} \right\}
\end{equation*}

Having computed the strong bisimulation equivalence, the next step in the reduction of the state space of the labeled transition system uses the bisimulation equivalence obtained in the first step in order to minimize the labeled graph. This reduction is implemented as follows:
\begin{enumerate}
	\item All states in a bisimilar equivalence class $B_{i}$ are merged into one single state $k=\bigcup p_{j}$, for $p_{j}\in B_{i}$;
	\item All incoming transitions $r \stackrel{a}{\rightarrow} p_{j}$, for $p_{j}\in B_{i}$, are replaced by transitions $r \stackrel{a}{\rightarrow} k$;
	\item All outgoing transitions $p_{j} \stackrel{a}{\rightarrow} t$, for $p_{j}\in B_{i}$, are replaced by transitions $k \stackrel{a}{\rightarrow} t$;
	\item The duplicate transitions are not taken into consideration.
\end{enumerate}
The procedure is repeated for all equivalence classes $B_{i}$, $i\in \overline{1,n}$.

This process of reduction with respect to strong bisimilarity is illustrated below in Fig.~\ref{fig:bisimGraph1}. The results obtained by applying both algorithms for computing strong bisimulation equivalence are given in Table~\ref{table1}. These results are then used as a basis for the reduction of the graph to its minimal form: all mutually bisimilar states are merged into a single state and their transitions are updated accordingly. 
\begin{figure}[h]
	\centering
	\includegraphics[width=3.0in]{bisimGraph1}
	\caption{Example of a labeled transition system graph and its minimial form modulo strong bisimilarity. The red dashed arrows depict the mapping of the states using the computed bisimulation equivalence. States 2 and 3 are merged into state 2 in the minimal graph, and states 4,5 and 6 are merged into state 3 in the minimal graph.}
	\label{fig:bisimGraph1}
\end{figure}

\begin{table}[h]
\begin{tabular}{| l | p{10.5cm}| }
  \hline                       
  Algorithm & Results \\ \hline
  Naive & (2, 3), (3, 2), (4, 5), 
(5, 4), (4, 6), (6, 4), (5, 6), (6, 5), (0, 0), (1, 1), (2, 2), (3, 3), (4, 4), (5, 5), (6, 6) \\ \hline
  Fernandez & \{0\}, \{1\}, \{2\}, \{3\}, \{4, 5, 6\} \\ \hline  
\end{tabular}
\\
\caption{Computing strong bisimularity for the example labeled graph from Fig.~\ref{fig:bisimGraph1}}
\label{table1}
\end{table}

\subsection{Minimization of a labeled transition system modulo weak bisimilarity}
The minimization of a labeled transition system modulo weak bisimilarity is reduced to the problem of minimization modulo strong bisimilarity, using a technique called saturation. Intuitevely, saturation amounts to first precomputing the weak transition relation and then constructing a new pair of finite processes whose original transitions are replaced by the weak transitions \cite{ReactiveSystems}. Once the algorithm for saturation is run and the original labeled transition system is saturated, the computation of weak bisimilarity amounts to computing strong bisimilarity over the saturated system.

The algorithm for saturation in TMACS was implemented as follows: 
\begin{enumerate}
\item For every ${p\in \mathit{S}}$, the set of all transitions $T$ of a labeled transition system with $m$ transitions and $n$ states is partitioned in ${2n}$ sets with: 
\begin{equation*}
	\begin{array}{lcl}
 		{T_{\mathit{\tau p}}=\left\{\left(p,\tau,q\right)| \left(p,\tau,q\right)\right\}}, \text{and}\\
    {T_{ap}=\left\{\left(p,a,q\right)| \left(p,a,q\right)\wedge a\neq\tau\right\}}
  \end{array}
\end{equation*} 
By the definition of ${T_{\mathit{\tau p}}}$ and ${T_{ap}}$, it follows that ${\bigcup_{p\in \mathit{S}}\left(T_{\mathit{\tau p}}\cup T_{ap}\right)=T}$, and also, that their pairwise intersection is empty. \\
The family of sets ${T^{'}_{\tau p}}$ is then iteratively constructed with:
\begin{equation*}
	\begin{array}{lcl}
		{T^{0}_{\mathit{\tau p}}=T_{\mathit{\tau p}}\cup\left\{\left(p,\tau,p\right)\right\}},\\
		{T^{i}_{\mathit{\tau p}}=T^{i-1}_{\mathit{\tau p}}\cup\left\{\left(p,\tau,q'\right)|\left(\exists q\in \mathit{S}\right)\left(p,\tau,q\right)\in T^{i-1}_{\mathit{\tau p}}\wedge\left(q,\tau,q'\right)\in T_{\mathit{\tau q}}\right\}}, \text{and} \\
		{T^{'}_{\mathit{\tau p}}=T^{n}_{\mathit{\tau p}}}
	\end{array}
\end{equation*}
(Note: ${\left|T^{'}_{\mathit{\tau p}}\right|\leq\left|\mathit{S}\right|=n}$; when for some ${k<n}$ it holds that ${T^{k}_{\mathit{\tau p}}=T^{k+1}_{\mathit{\tau p}}}$, then ${T^{'}_{\mathit{\tau p}}=T^{k}_{\mathit{\tau p}}=T^{n}_{\mathit{\tau p}}}$)\\

With this step a reflexive, transitive closure of $\tau$ is constructed:
\begin{equation*}
	{T^{*}_{\tau}=\bigcup_{p\in \mathit{S}}T^{'}_{\tau p}=\left\{\left(p,\tau,q\right)|\hspace{1mm}p\left(\stackrel{\tau}{\rightarrow}\right)^{*}q\right\}}
\end{equation*}

An example of a reflexive, transitive closure of $\tau$ as computed in this step, is shown on Fig.~\ref{fig:saturation}.\\

\begin{figure}[h]
\centering
\includegraphics[width=4.5in]{saturation}
\caption{Reflexive, transitive closure of $\tau$. The original graph is depicted with red lines.}
\label{fig:saturation}
\end{figure}

\item The next step is to construct
\begin{equation*}\label{eq:tap}
		T'_{s}=\bigcup_{p\in \mathit{S}}T'_{ap}=\left\{\left(p,a,q\right)|\left(\exists q'\in \mathit{S}\right)\left(p,a,q'\right)\in T\wedge q'\left(\stackrel{\tau}{\rightarrow}\right)^{*}q\right\}
\end{equation*}
as follows:
\begin{equation*}
	\begin{array}{lcl}
		T^{0}_{sp}=T_{sp},\\
		T^{i}_{sp}=T^{i-1}_{sp}\cup \left\{\left(p,a,q'\right)|\left(\exists q\in \mathit{S}\right)\left(p,a,q\right)\in T^{i-1}_{sp}\wedge \left(q,\tau,q'\right)\in T^{i-1}_{\tau q}\right\} \text{and} \\
		T'_{sp}=T^{n|\mathit{Act}|}_{sp}
	\end{array}
\end{equation*}

(Note: $|T'_{sp}|\leq |\mathit{S}||\mathit{Act}| = n|\mathit{Act}|$, and when for some $k < n|\mathit{Act}|$ it holds that $T^{k}_{sp}=T^{k+1}_{sp}$, then $T'_{sp}=T^{k}_{sp}=T^{n|\mathit{Act}|}_{sp}$)\\

\item For the third step, $T'=\bigcup_{p\in \mathit{S}}\left(T^{'}_{\mathit{\tau p}}\cup T'_{sp}\right)$ needs to be partitioned again, defined by the destination in the transition triple:
\begin{equation*}
	\begin{array}{lcl}
		T^{*}_{\mathit{\tau q}}=\left\{\left(p,\tau,q\right)|\left(p,\tau,q\right)\in T'\right\}, \text{and}\\
		T_{dq}=\left\{\left(p,a,q\right)|\hspace{1mm}\left(p,a,q\right)\in T' a\neq\tau\right\}				    
	\end{array}
\end{equation*}
for every $p\in \mathit{S}$, and then construct:
\begin{equation*}
	\begin{array}{lcl}
		T^{0}_{dq}=T_{dq},\\
		T^{i}_{dq}=T^{i-1}_{dq}\cup\left\{\left(p',a,q\right)|\left(\exists p\in \mathit{S}\right)\left(p,a,q\right)\in T^{i-1}_{dq}\wedge\left(p',\tau,p\right)\in T^{*}_{\tau p}\right\} \text{, and}\\
		T^{*}_{dq}=T^{n|\mathit{Act}|}_{dq}
	\end{array}
\end{equation*}
\end{enumerate}

Finally the saturated labeled transition system now is:
\begin{align*}
	T^{*} &=\bigcup_{p\in \mathit{S}}\left(T^{*}_{\mathit{\tau p}}\cup T^{*}_{dp}\right)=\\
	&=\left(\stackrel{\tau}{\rightarrow}\right)^{*}\cup\left\{\left(p,a,q\right)|\hspace{1mm}a\neq\tau\wedge\left(\exists p',q'\in S\right)\hspace{1mm} p\left(\stackrel{\tau}{\rightarrow}\right)^{*}p'\stackrel{a}{\rightarrow}q'\left(\stackrel{\tau}{\rightarrow}\right)^{*}q\right\}
\end{align*}
An illustration of a saturated labeled transition system is given in Fig.~\ref{fig:saturation2} for the original labeled transition system Fig.~\ref{fig:saturation3}:

\begin{figure}[h]
\centering
\includegraphics[width=3.5in]{saturation3}
\caption{Example of a labeled graph before saturation}
\label{fig:saturation3}
\end{figure}
\begin{figure}[h]
\centering
\includegraphics[width=4.5in]{saturation2}
\caption{The labeled graph from Fig.~\ref{fig:saturation3} after saturation}
\label{fig:saturation2}
\end{figure}

Having computed the observational equivalence (weak bisimilarity) of the original labeled transition system, the process of its minimization is the same as the process for minimization modulo strong bisimilarity applied on the saturated labeled transition system.

\subsection{Comparison of two labeled transition systems modulo strong bisimilarity}
The idea for the implementation of the equivalence checking of two labeled transition systems modulo strong bisimilarity was based on the following fact: Two labelled transition systems are (strongly) bisimilar iff their initial states are bisimilar \cite{ModellingAndAnalysis}. That means that in order to check whether two labeled transition systems are bisimilar it is enough to check whether their initial states are bisimilar. This can be done using the following approach:
\begin{enumerate}
	\item The two labeled transition systems are merged into a single transition system
	\item An algorithm for computing the strong bisimilarity is applied to the merged system
	\item A check is performed to see if the initial states belong to the same bisimulation equivalence class
\end{enumerate}

\subsection{Comparison of two labeled transition systems modulo weak bisimilarity}
The comparison of two labeled transition systems modulo weak bisimilariy amounts to checking strong bisimilarity over the saturated labeled transition systems \cite{ReactiveSystems}. In another words, two labeled transition systems are weakly bisimilar iff their saturated labeled transition systems are strongly bisimilar. Following this fact, we implemented the comparison of two labeled transition systems modulo weak bisimilarity by applying the saturation algorithm over the original labeled graphs in order to obtain their saturated labeled graphs, after which the process of comparison of the saturated labeled transition systems modulo strong bisimilarity was applied as described above. % Chapter4
\section{$U^w$ and $U^s$ implementation}

This part explains how implementation of Hennessy-Milner logic works. For a given Hennessy-Milner expression and a graph, we should get an output whether that expression is valid for that graph.  One Hennessy-Milner expression can be made of these set of tokens \{"AND", "OR", "UW", "US", "NOT", "[", "]", "$\langle$", "$\rangle$", "TT", "FF", "(", ")", "{", "}", ","\}.  The first part of this process is called tokenization. Tokenization is the process of demarcating and possibly classifying sections of a string of input characters. The resulting tokens are then passed on to some other form of processing.
The process can be considered as a sub-task of parsing input. As the tokens are being read they are proceeded to the parser. The parser is a LR top-down parser. This parser is able to recognize a non-deterministic grammar which is essential for the evaluation of the Hennessy-Milner expression. As the parser is reading the tokens, thus in a way we “move” through the graph and determine if some of the following steps are possible. Every condition is in a way kept on stack and that enables later return to any of the previous conditions. The process is finished when the expression  is processed or when it is in a condition from which none of the following actions can be taken over.

The implementing of these operators is made in this way: we get the current state. Current state is the state which contains all the nodes that satisfied the previous actions. For example if we have expression $\langle$a$\rangle$ $\langle$a$\rangle$TT $U^s$ $\langle$b$\rangle$TT, than we get all the nodes that we can reach, starting from the start node, with two actions “a”. When we get to the “Until” operation, in this case $U^s$, we take the start state and check if we can make an action b from some nod. If we can’t do action b, than we repeat the process, do a again and check if can we do b. This process is being repeated until we come to a state in which we can't do action a from all nodes in that state.
In the end we check the operator’s type (strong or weak). If the operator is strong, than only one node in its state is enough. If it is weak, we check whether in its new state there aren’t any nodes or if from the starting node we can get to some of its neighbors through action b. \\

BNF\\\\
HML =$\rangle$ TT $\vert$ FF $\vert$ HML UW HML $\vert$ HML US HML $\vert$ HML AND HML $\vert$ HML OR HML $\vert$ $\langle$a$\rangle$HML $\vert$ [a]HML $\vert$ (HML) $\vert$ NOT HML
\\\\
Grammar\\
\begin{enumerate}
\item HML =$\rangle$ Term Expression
\item Expression =$\rangle$ AND Term Expression
\item Expression =$\rangle$ OR Term Expression
\item Expression =$\rangle$ Uw Term Expression
\item Expression =$\rangle$ Us Term Expression
\item Expression =$\rangle$ $\lambda$
\item Term =$\rangle$ NOT Term
\item Term =$\rangle$ [ Actions ] Term
\item Term =$\rangle$ $\langle$ Actions $\rangle$ Term
\item Term =$\rangle$ TT
\item Term =$\rangle$ FF
\item Term =$\rangle$ ( HML )
\item Actions =$\rangle$ Action
\item Actions =$\rangle$ { Action ActionsList }
\item ActionsList =$\rangle$ , Action ActionsList
\item ActionsList =$\rangle$ $\lambda$
\item Action =$\rangle$ Literal Name
\item Name =$\rangle$ Literal
\item Name =$\rangle$ Number
\item Name =$\rangle$ $\lambda$
\item Literal =$\rangle$ a $\vert$ b $\vert$ c $\vert$ d $\vert$ e $\vert$ f $\vert$ g $\vert$ h $\vert$ i $\vert$ j $\vert$ k $\vert$ l $\vert$ m $\vert$ n $\vert$ o $\vert$ p $\vert$ q $\vert$ r $\vert$ s $\vert$ t $\vert$ u $\vert$ v $\vert$ w $\vert$ x $\vert$ y $\vert$ z
\item Number =$\rangle$ 0 $\vert$ 1 $\vert$ 2 $\vert$ 3 $\vert$ 4 $\vert$ 5 $\vert$ 6 $\vert$ 7 $\vert$ 8 $\vert$ 9
\end{enumerate}


 % Chapter5
\section{Application}
\label{sec:application}
%TMACS is still in early development phase and it has the functionality needed to perform modeling, formal specification, and verification of concurrent systems. 
In this chapter we demonstrate that process of formal specification and verification for 
%two classical problems in the concurrency theory: the alternating bit protocol and the mutual exclusion algorithm of Peterson.
one classical problem: the alternating bit protocol.

%\subsection{Alternating bit protocol - modelling, specification and verification}

\subsubsection{Modelling and Specification.}
The representation of the Alternating Bit Protocol consists of a $sender$ $S$, a $receiver$ $R$ and two channels $transport$ $T$ and $acknowledge$ $A$ as shown on Fig.~\ref{fig:abp}. 

\begin{figure}[h]
\centering
\includegraphics[width=3.5in]{abp}
\caption{Alternating Bit Protocol representation}
\label{fig:abp}
\end{figure}

The only visible transitions in the alternating bit protocol are $deliver$ and $accept$, which can occur only sequentially, whereas all others are internal synchronizations. Sender $S$ sends a message which contains the protocol bit, 0 or 1, to a receiver $R$. The channel from $S$ to $R$ is initialized and there are no messages in transit. There is no direct communication between the sender $S$ and the receiver $R$, and all messages must travel trough the medium ($transport$ and $acknowledge$ channel). 

The functioning of the alternating bit protocol can be described as \cite{ReactiveSystems}:
\begin{enumerate}
	\item The sender $S$ sends a message repeatedly (with its corresponding bit) until it receives an acknowledgment ($ack0$ or $ack1$) from the receiver $R$ that contains the same protocol bit as the message being sent. This behaviour of the process representing the sender can be described as:
				\begin{equation*}\label{send_imp}
				    \begin{array}{lcl}
							S = \overline{send0}.S+ack0.accept.S_{1}+ack1.S \\
							S_{1}=\overline{send1}.S_{1}+ack1.accept.S+ack0.S_{1}				  
						\end{array}
				\end{equation*}
	      The transport channel transmits the message to the receiver, but it may lose the message (lossy channel) or transmit it several times (chatty channel). Therefore, the description of the behaviour of the process representing the transport channel is given with CCS expression as follows:
	      \begin{equation*}\label{trans_imp}
	      	\begin{array}{lcl}
						T=send0.\left(T+T_{1}\right)+send1.\left(T+T_{2}\right)\\
						T_{1}=\overline{receive0}.\left(T+T_{1}\right)\\
						T_{2}=\overline{receive1}.\left(T+T_{2}\right)
					\end{array}
				\end{equation*}
  \item When the receiver $R$ receives a message, it sends a reply to $S$ which includes the protocol bit of the message received. When the message is received for the first time, the receiver will deliver it for processing, while subsequent messages with the same bit will be simply acknowledged. That yields the following CCS expression for the receiver:
  			\begin{equation*}\label{rec_imp}
				    \begin{array}{lcl}
							R=receive0.\overline{deliver}.R_{1}+\overline{reply1}.R+receive1.R \\
							R_{1}=receive1.\overline{deliver}.R+\overline{reply0}.R_{1}+receive0.R_{1}			  
						\end{array}
				\end{equation*}
        Again, the acknowledgement channel sends the $ack$ to sender, and it can also acknowledge it several times or lose it on the way to the sender. Therefore the ackowledgement channel and its behaviour is described as follows:
        \begin{equation*}\label{trans_imp}
	      	\begin{array}{lcl}
						A=reply0.\left(A+A_{1}\right)+reply1.\left(A+A_{2}\right)\\
						A_{1}=\overline{ack0}.\left(A+A_{1}\right)\\
						A_{2}=\overline{ack1}.\left(A+A_{2}\right)
					\end{array}
				\end{equation*}
  \item When $S$ receives an acknowledgment containing the same bit as the message it is currently transmitting, it stops transmitting that message, flips the protocol bit, and repeats the protocol for the next message.\cite{Kulick}\cite{ProcessAlgebraParallel}
\end{enumerate}

Having described the behaviour of the alternating bit protocol components, the CCS process expression describing the behaviour of the protocol as a whole can be obtained as a parallel composition of the processes describing the sender, the transport channel, the receiver and the acknowledgement channel:
\begin{equation}\label{abp_imp}
	\mathit{ABP} \stackrel{def}{=}\left(S|T|R|A\right)\backslash L,
\end{equation}
restricted on the set of actions:
\begin{equation*}
  L = \left(send0,send1,receive0,receive1,reply0,reply1,ack0,ack1\right)
\end{equation*}

This CCS expression represents the implementation of the alternating bit protocol which details the proposed means for achieving the desired high-level behaviour the alternating bit protocol should exhibit. This desired high-level behaviour is that the alternating bit protocol should act as a simple buffer, therefore its CCS specification is defined as follows:
\begin{equation}\label{eq:abp_spec}
	\begin{array}{lcl}
		\mathit{Buf} = \mathit{accept.Buf'}\\
		\mathit{Buf'} = \mathit{\overline{deliver}.Buf}
	\end{array}
\end{equation}

\subsubsection{Verification.} In order to verify the alternating bit protocol, we need to prove that the implementation $ABP$ meets the specification $Buf$ with respect to some behavioural equivalence. We shall show that an observational equivalence between $\mathit{Buf}$ and $\mathit{ABP}$ can be found, i.e. that $\mathit{ABP}\approx \mathit{Imp}$. For that purpose, first we use TMACS to obtain the labeled graphs corresponding to the CCS representations of $\mathit{Buf}$ and $\mathit{ABP}$, and afterwards we perform a comparison of the labeled transition systems modulo weak bisimilarity which yields a positive answer about the existance of weak bisimulation equivalence between $\mathit{Buf}$ and $\mathit{ABP}$.

The weak bisimulation equivalence obtained by running any of the two bisimulation algorithms implemented in TMACS over the saturated labeled transition systems is given in Table~\ref{table3}:

\begin{table}
\begin{tabular}{| p{7.5cm} | p{4.5cm} | }
	
  \hline                       
	ABP implementation states &
	ABP specification states
	\\ \hline
	
$\left(S|T|R|A\right)\backslash L$ & \\
$\left(S|\left(T+T1\right)|R|A\right)\backslash L$ & \\
$\left(S|\left(T+T1\right)|R|\left(A+A2\right)\right)\backslash L$ & \\
$\left(S|T|R|\left(A+A2\right)\right)\backslash L$ & \\
$\left(S|\left(T+T1\right)|\overline{deliver}.R1|A\right)\backslash L$ & \\
$\left(S|\left(T+T1\right)|\overline{deliver}.R1|\left(A+A2\right)\right)\backslash L$ & \\
$\left(S1|\left(T+T1\right)|R1|\left(A+A1\right)\right)\backslash L$ & \\
$\left(S1|\left(T+T2\right)|\overline{deliver}.R|\left(A+A1\right)\right)\backslash L$ & \\
$\left(S1|\left(T+T2\right)|R1|\left(A+A1\right)\right)\backslash L$ & \\
$\left(S|\left(T+T2\right)|R|\left(A+A2\right)\right)\backslash L$ &
  $\mathit{Buf}$   
  \\ \hline
   
$\left(accept.S1|\left(T+T1\right)|R1|\left(A+A1\right)\right)\backslash L$ & \\ 
$\left(S|\left(T+T1\right)|R1|\left(A+A1\right)\right)\backslash L$ & \\
$\left(S|\left(T+T1\right)|R1|\left(A+A2\right)\right)\backslash L$ & \\
$\left(S|\left(T+T1\right)|R1|A\right)\backslash L$ & \\
$\left(S1|\left(T+T2\right)|R|\left(A+A2\right)\right)\backslash L$ & \\
$\left(accept.S|\left(T+T2\right)|R|\left(A+A2\right)\right)\backslash L$ & \\
$\left(S1|\left(T+T2\right)|R|\left(A+A1\right)\right)\backslash L$ &
  $\mathit{Buf'}$
  \\ \hline  
\end{tabular}

\caption{Verification of the alternating bit protocol using weak bisimilarity}
\label{table3}
\end{table}

%\subsection{Peterson's algorithm - Modeling,  Specification and Testing}

Peterson's algorithm \cite{Peterson} is a simple algorithm designed to ensure mutual exclusion between two processes without any special hardware support. It represents a simple refinement of ideas from earlier mutex algorithms such as Dekker's algorithm \cite{Dekker}. Mutual exclusion (often abbreviated as mutex) algorithms are used in concurrent programming to avoid the simultaneous use of a common resource by critical sections. A critical section is a piece of code in which a process or thread accesses a common resource. 

\subsubsection{Modeling and Specification.}In Peterson's algorithm for mutual exclusion, there are:
\begin{itemize}
  \item Two processes $P_{1}$ and $P_{2}$ that want to access the same resource, i.e. eneter the critical section;
	\item Two shared variables $b_{1}$ and $b_{2}$ which indicate whether process $P_{1}$ and process $P_{2}$ are trying to enter the critical section;
	\item A shared integer variable $k$ that can take one of the values 1 or 2 and indicates which process is next to enter the critical section;
\end{itemize}

The boolean variables $b_{1}$ and $b_{2}$ are initialized to values $'false'$ because neither of the processes is interested yet to enter the critical section, whereas the initial value of $k$ is arbitrary. 

In order to ensure mutual exclusion, each process $P_{i}$, $i\in\left\{1,2\right\}$, executes the following algorithm presented in pseudocode, where $j$ denotes the index of the other process. 

\begin{algorithm}
\caption{Peterson's algorithm pseudocode}
\begin{algorithmic}
\WHILE{$true$} 
	\STATE '<noncritical section>';
	\STATE $b_{i} \gets true$;
	\STATE $k \gets j$;
	\WHILE {$\left(b_{j} and k = j\right)$}
		 \STATE skip;
	\ENDWHILE
	\STATE '<critical section>';
	\STATE $b_{i} \gets false$;
\ENDWHILE
\end{algorithmic}
\end{algorithm}

The desired behaviour of the Peterson's algorithm is as the one of any simple mutex algorithm. Initially, both processes enter their critical sections, however once one of the processes has entered its critical section, the other process cannot enter its own critical section until the first process has exited its critical section. Therefore, a suitable CCS specification of the behaviour of a mutual exclusion algorithm like Peterson's is given as follows:
\begin{equation}
	MutexSpec = enter1.exit1.MutexSpec + enter2.exit2.MutexSpec
\end{equation}

Modeling the algorithm of Peterson includes, among other tasks, translation of the algorithm's pseudocode description of the behaviour of the processes $P_{1}$ and $P_{2}$ into the model of CCS or LTSs. 

Following the message-passing paradigm on which CCS is based, the variables manipulated by the processes $P_{1}$ and $P_{2}$ are viewed as passive agents that react to actions performed by the processes. The description of the variables used in Peterson's algorithm as processes can be done as follows:
\begin{enumerate}
	\item The process representing the shared boolean variable $b_{1}$ has two states and its behaviour can be represented by the following CCS expressions:\\
				\begin{equation*}\label{b1_imp}
				    \begin{array}{lcl}
							B1f = \overline{b1rf}.B1f + b1wf.B1f + b1wt.B1t \\
							B1t = \overline{b1rt}.B1t + b1wf.B1f + b1wt.B1t			  
						\end{array}
				\end{equation*}
	      Similarly for the process describing the behaviour of the variable $b_{2}$:\\
	      \begin{equation*}\label{b2_imp}
				    \begin{array}{lcl}
							B2f = \overline{b2rf}.B2f + b2wf.B2f + b2wt.B2t \\
							B2t = \overline{b2rt}.B2t + b2wf.B2f + b2wt.B2t,		  
						\end{array}
				\end{equation*}
	      where the pattern for the channel name is $b<i><x><y>$, with
	      \begin{itemize}
					\item $i\in\left\{1,2\right\}$ for the process ID
					\item $x\in\left\{r,w\right\}$ for the type of operation (read or write)
					\item $y\in\left\{f,t\right\}$ for the variable value to be written or read (false or true)
				\end{itemize}
	\item The process representing the variable $k$ has two states, denoted by the constants $K_{1}$ and $K_{2}$, because the variable $k$ can only take one of the two values 1 and 2, and its CCS representation is as follows\\
				\begin{equation*}\label{k_imp}
				    \begin{array}{lcl}
							K1 = \overline{kr1}.K1 + kw1.K1 + kw2.K2 \\
							K2 = \overline{kr2}.K2 + kw2.K2 + kw2.K2,		  
						\end{array}
				\end{equation*}
				where the pattern for the channel name is $k<x><n>$, with
			  \begin{itemize}
					\item $x\in\left\{r,w\right\}$ for the type of operation (read or write)
					\item $n\in\left\{1,2\right\}$ for the value to be written or read
				\end{itemize}
\end{enumerate}

The final step is the CCS formalisation of the behaviour of the processes $P_{1}$ and $P_{2}$. The process behaviour outside of the critical region can be ignored and the focus can be put on the process entering and exiting the critical section. For simplicity, an assumption is made that the processes cannot fail or terminate within the critical section. Under these assumptions, the initial behaviour of the process $P_{1}$ can be described by the following CCS expression:\\
				\begin{equation*}\label{p1_imp}
				    P1 = \overline{b1wt}.\overline{kw2}.P11,
				\end{equation*}
				where $P11$ models the while loop (with short-circuit evaluation):
				\begin{equation*}\label{p11_imp}
				    P11 = b2rf.P12 + b2rt.\left(kr2.P11 + kr1.P12\right)
				\end{equation*}
				and $P12$ models the critical section:
				\begin{equation*}\label{p12_imp}
				    P12 = enter1.exit1.\overline{b1wf}.P1
				\end{equation*}

The behaviour of the process $P_{2}$ can be described with CCS expressions in the similar manner:\\
				\begin{equation*}\label{p2_imp}
				    \begin{array}{lcl}
							P2 = \overline{b2wt}.\overline{kw1}.P21 \\
							P21 = b1rf.P22 + b1rt.\left(kr1.P21 + kr2.P22\right)\\
							P22 = enter2.exit2.\overline{b2wf}.P2	  
						\end{array}
				\end{equation*}

Finally, the CCS process term representing the whole Peterson's algorithm consists of the parallel composition of the terms describing the two processing running the algorithm and of those describing the variables. Since we are only interested in the behaviour of the algorithm pertaining to the access to, and exit from, their critical sections, we shall restrict all the communication channels that are used to read from, and write to, the variables. That set of restricted channel names is 
\begin{equation*}
	L = \left\{b1rf,b1rt,b1wf,b1wt,b2rf,b2rt,b2wf,b2wt,kr1,kw1,kr2,kw2\right\}
\end{equation*}

Assuming that the initial value of the variable $k$ is 1, our CCS description of Peterson�s algorithm is therefore given by the term:

\begin{equation}\label{pet_imp_full}
	PETERSON \stackrel{def}{=} \left(B1f|B2f|K1|P1|P2\right)\backslash L
\end{equation}

\subsubsection{Testing.} Testing the preservation of the mutual exclusion property is one of the approaches that can be used to establish the correctness of the Peterson's algorithm. A test is a finite rooted LTS over the set of actions $Act\cup\left\{\overline{bad}\right\}$, where $bad$ is a distinguished channel name not occurring in $Act$. The idea is that the test would act as a monitor process that 'observes' the behaviour of a process and reports an error in case of an occurrence of an undesirable situation by performing a $bad$-labelled transition. Assuming that the monitor process outputs 'bad' when it discovers that two enter actions have occurred without intervening exit, a CCS process describing this behaviour is:
\begin{equation*}\label{mut_test}
  \begin{array}{lcl}
  	MutexTest = \overline{enter1}.MutexTest1 + \overline{enter2}.MutexTest2 \\
		MutexTest1 = \overline{exit1}.MutexTest + \overline{enter2}.\overline{bad}.0\\
		MutexTest2 = \overline{exit2}.MutexTest + \overline{exit1}.\overline{bad}.0
	\end{array}
\end{equation*}

Now, in order to check whether process $PETERSON$ ensures mutual exclusion, it is now sufficient to let it interact with $MutexTest$ and see if the resulting system
\begin{equation}\label{pet_test}
	\left(PETERSON|MutexTest\right)\backslash M,
\end{equation}
where 
\begin{equation*}
	\left\{enter1,enter2,exit1,exit2\right\},
\end{equation*}
can initially perform the action $\overline{bad}$.

Indeed, the LTS of the CCS expression (\ref{pet_test}) generated by our tool does not have states which afford $bad$ transitions. This proves that Peterson's algorithm ensures mutual exclusion. % Chapter6
%\section{Bisimulation equivalence}
% no \IEEEPARstart
Bisimulation equivalence (bisimilarity)\footnote{The notion of bisimulation equivalence (bisimilarity) in this chapter 
refers to strong bisimulation equivalence (strong bisimilarity)} is a binary relation between labeled transition systems 
which associates systems that can simulate each other's behaviour in a stepwise manner. This enables comparison of 
different transition systems.

An alternative perspective is to consider bisimulation equivalence as a relation between states of a single labelled 
transition system. By considering the quotient transition system under such a relation, smaller models are obtained
[BK08].

The bisimulation equivalence finds its extensive application in the area of formal verification of concurrent systems,
for example to check the equivalence of an implementation of a certain system with respect to its specification model.

In our tool the process of determining an existance of a bisimulation equivalence 
between two labeled transition systems was implemented using an approach which consists of three steps:

\begin{enumerate}
\item Computing strong bisimulation equivalence (strong bisimilarity) for each of the two LTSs;
\item Minimizing each of the two LTSs to its canonical form using the strong bisimilarity obtained
in the first step;
\item Performing a comparison between the two canonical forms obtained in the second step.
\end{enumerate}

The first step, computing strong bisimulation equivalence, was implemented with two different methods: the so called
naive method and a more efficient method due to Fernandez, both of which can serve as minimization procedures.

The naive algorithm [AILS07a] for computing bisimulation equivalence stems from the theory underlying Tarski's fixed point
theorem [AILS07b]. It has been proven that the strong bisimulation equivalence is the largest fixed point of the 
monotic function $F$ as defined in [AILS07a] given by Tarsky's fixed point theorem. 

This algorithm has time complexity of $O(mn)$ for a labeled transition system with
\emph{m} transitions and \emph{n} states. 

In our implementation the algorithm takes as input an LTS in aldebaran format, generates a corresponding labeled 
graph and then computes the strong bisimulation equivalence as pairs of bisimilar states.

The algorithm due to Fernandez exploits the idea of the relationship between strong bisimulation equivalence 
and the relational coarsest partition problem solved by Paige and Tarjan. It represents adaptation of the 
Paige-Tarjan algorithm of complexity $O(m \log n)$ to minimize labeled transition systems modulo bisimulation 
equivalence by computing the coarsest partition problem with respect to the family of binary relations 
$\left(T_a\right)_{a\in A}$ instead of one binary relation, where $T_a=\{(p,q)|(p,a,q)\in T\}$ is a transition 
relation for action ${a\in A}$ and $T$ is a set of all transitions [PT87, Fer89].

The algorithm due to Fernandez in our implementation takes an LTS in aldebaran format as an input, generates a 
corresponding labeled graph and then partitions the labeled graph into its coarsest blocks where each block represents 
a set of bisimilar states. Partition is a set of mutually exclusive blocks whose union constitutes the graph universe.

To define graph transitions the following terminology was used: 

\begin{itemize}
	\item $T_a[p]=\{q\}$ - an $a$-transition from state $p$ to state $q$
	\item $T_a{}^{-1}[q]=\{p\}$ - an inverse $a$-transition from state $q$ to state $p$
	\item $T_a{}^{-1}[B]=\cup \left\{T_a{}^{-1}[q],q\in B\right\}$ - inverse transition for block $B$ and action $a$
	\item $W$ - set of sets called splitters that are being used to split the partition
	\item infoB$(a, p)$ - info map for block $B$, state $p$ and action $a$
\end{itemize}

The time complexity of Fernandez's algorithm is $O(m \log n)$ for a labeled transition system 
with $m$ transitions and $n$ states. 

Both algorithms were implemented in Java. Several different data structures were used to implement the structure
labelled graph. The labelled graph is represented as a list of nodes. Each node is represented by the number of the
corresponding state, as a start state, and a list whose elements are couples of an outgoing action and a reachable 
state. 

% Sp = {(a, q)|a pripagja na A, q pripagja na Q, p ->a q}

% The algorithm of Fernandez requires additional data structures to represent the Ta, Ta_inversno (kako mnozestvo
% od trudot na Fernandez).

The next step uses the bisimulation equivalence computed in the first step in order to minimize the graphs. This reduction 
is implemented as follows:
\begin{enumerate}
	\item If a pair of states $(p, q)$ is bisimilar, then the two states are merged into one single state $k$;
	\item All incoming transitions $r \stackrel{a}{\rightarrow} p$ and $s \stackrel{a}{\rightarrow} q$ are replaced by transitions $r \stackrel{a}{\rightarrow} k$ and $s \stackrel{a}{\rightarrow} k$;
	\item All outgoing transitions $p \stackrel{a}{\rightarrow} r$ and $q \stackrel{a}{\rightarrow} s$ are replaced by transitions $k \stackrel{a}{\rightarrow} r$ and $k \stackrel{a}{\rightarrow} s$;
	\item The duplicate transitions are not taken into consideration.
\end{enumerate}
The procedure is repeated for all pairs of bisimilar states.

Having reduced the two labeled graphs into their minimal forms, the last step in the process of checking the equivalence
between the two labeled transition system consists of checking whether the two minimal labeled graphs are isomorphic.

Two graphs $G$ and $H$ are isomorphic if there exists a graph isomorphism between them. According to [BPS01], a graph isomorphism 
between two graphs G and H is a bijective function $f: Nodes(G) \rightarrow Nodes(H)$ satisfying:
\begin{itemize}
	\item $f(initialNode(G)) = g(initialNode(H))$
	\item $(s, a, t)$ is an edge in $G \Leftrightarrow (f(s), a, f(t))$ is an edge in $H$
\end{itemize}

The graph isomorphism check is implemented as follows:
Starting from the initial states, the outgoing tranzitions have to be the same, 
Povtorno pocnuvate od pocetnite sostojbi. Izleznite tranzicii mora da bidat isti, i za sekoja izlezna tranzicija so nekakva oznaka, mora da postoi druga takva so ista oznaka vo drugiot graf sto vodi povtorno do izomorfni sostojbi. Znaci treba da gi cuvate za sekoja sostojba nejzinite opcii za mozni izomorfni sostojbi. Posetenite sostojbi ne gi proveruvate (znaci ova e ednostavna rekurzija).

The two graphs are bisimilar.

The correctness of the implementation was tested with the use of ltsconvert and ltscompare tools of MCRL2, ....[ref6]

The whole process as described above is illustrated on the examples in Fig 

\begin{figure}[h!]
\centering
\includegraphics[width=2.0in]{bisimGraph2}
\caption{Minimized Graph 2}
\label{fig:bisimGraph2}
\end{figure}

\begin{figure}[h!]
\centering
\subfigure[Graph 1]{
\includegraphics[width=2.3in]{graph1}
\label{fig:graph1}
}
\subfigure[Graph 2]{
\includegraphics[width=0.9in]{graph2}
\label{fig:graph2}
}
\label{fig:exampleGraphs}
\caption{Example of two labeled graphs}
\end{figure}

\begin{figure}[h!]
\centering
\includegraphics[width=3.0in]{bisimGraph1}
\caption{Minimized Graph 1}
\label{fig:bisimGraph1}
\end{figure}

An illustration of the output of the algorithms for the graphs shown on Fig. 2 and Fig. 3 is given in TABLE I.

\begin{table}[h!]
\begin{tabular}{| l | p{3.2cm}| p{3.2cm} | }
  \hline                       
  Algorithm & Graph 1 & Graph 2 \\ \hline
  Naive & \{(2, 3), (3, 2), (4, 5), 
(5, 4), (4, 6), (6, 4), (5, 6), (6, 5), (0, 0), (1, 1), (2, 2), (3, 3), (4, 4), (5, 5), (6, 6)\} & \{(3, 4), (4, 3), (0, 0), 
(1, 1), (2, 2), (3, 3), (4, 4)\} \\ \hline
  Fernandez & \{\{0\}, \{1\}, \{2\}, \{3\}, \{4, 5, 6\}\} & \{\{0\}, \{1\}, \{2\}, \{3, 4\}\} \\ \hline  
\end{tabular}
\caption{Computing strong bisimularity for Graph 1 and Graph 2}
\label{table1}
\end{table}

The process of reduction of the Graph 1 to its minimal form is given on Fig. 4. As it can be seen from the figure, the states 
2 and 3 are merged into state 2 in the minimal graph, and states 4, 5, and 6 are merged into state 3 in the minimal graph.
\begin{figure}[h!]
\centering
\includegraphics[width=3.0in]{bisimGraph1}
% where an .eps filename suffix will be assumed under latex, 
% and a .pdf suffix will be assumed for pdflatex; or what has been declared
% via \DeclareGraphicsExtensions.
\caption{Minimized Graph 1}
\label{fig:bisimGraph1}
\end{figure}

The process of reduction of the Graph 2 to its minimal form is given on Fig. 5. As before, the states 
3 and 4 are merged into state 3 in the minimal graph.
\begin{figure}[h!]
\centering
\includegraphics[width=1.8in]{bisimGraph2}
% where an .eps filename suffix will be assumed under latex, 
% and a .pdf suffix will be assumed for pdflatex; or what has been declared
% via \DeclareGraphicsExtensions.
\caption{Minimized Graph 2}
\label{fig:bisimGraph2}
\end{figure}

% [ref6}: url za mcrl2
% [ref7]: Model checking 
%\section{Saturation and weak bisimulation equivalence}
% no \IEEEPARstart
Def: Let $P$ and $Q$ be CCS processes or, more generally, states in a labeled transition system. For each action $a$, we shall write
$P\stackrel{a}{\Rightarrow}Q$ iff:
\begin{itemize}
	\item Either $a\neq\tau$ and there are processes $P'$ and $Q'$ such that $P\left(\stackrel{\tau}{\rightarrow}\right)^{*}P'\ \stackrel{a}{\rightarrow}\ Q'\left(\stackrel{\tau}{\rightarrow}\right)^{*}Q$,
	\item Or $a=\tau}$ and ${P\left(\stackrel{\tau}{\rightarrow}\right)^{*}Q$,
\end{itemize}
where we write $\left(\stackrel{\tau}{\rightarrow}\right)^{*}$ for he reflexive and transitive closure of the relation $\stackrel{\tau}{\rightarrow}$.

Def: (weak bisimulation and observational equivalence) A binary relation $R$ over the set of states of an LTS is a weak bisimulation iff, whenever $s_{1}Rs_{2}$ and $a$ is an action (including $\tau$):
\begin{itemize}
	\item If $s_{1}\stackrel{a}{\rightarrow}s_{1}'$ then there is a transition $s_{2}\stackrel{a}{\Rightarrow}s_{2}'$ such that $s_{1}'Rs_{2}'$;
	\item If $s_{2}\stackrel{a}{\rightarrow}s_{2}'$ then there is a transition $s_{1}\stackrel{a}{\Rightarrow}s_{1}'$ such that $s_{1}'Rs_{2}'$;
\end{itemize}

Two states $s$ and $s'$ are observationally equivalent (or weakly bisimilar), written $s\approx s'$, iff there is a weak bisimulation equivalence that relates them.

Def: Let $T\subseteq S\times Act\times S$ be an LTS. We shall say that\\ 
$T^{*}=\left\{\left(p,a,q\right)| p\stackrel{a}{\Rightarrow}q\right\}=T\cup\left(\stackrel{a}{\rightarrow}\right)^{*}\cup\left\{\left(p,a,q\right)| a\neq\tau\wedge\left(\exists p',q'\in A\right) p\left(\stackrel{\tau}{\rightarrow}\right)^{*}p'\stackrel{a}{\rightarrow}q'\left(\stackrel{\tau}{\rightarrow}\right)^{*}q\right\}$ is a saturation of T.

Proposition: Two LTSs are weakly bisimilar iff their saturated LTSs are strongly bisimilar. 

Proof: Let $T$ and $U$ be two LTSs for which their saturated LTSs are strongly bisimilar by the relation $R$. Since the strong bisimulation is also a weak bisimulation it follows that $T$ and $U$ are weakly bisimular, since they are weakly bisimular to their respective saturated LTSs.
Let $T$ and $U$ be two LTSs that are weakly bisimilar. Let $u\approx t$, $u'\approx t'$. If $t\stackrel{a}{\rightarrow}t'\in T^{*}$, that means $t\stackrel{a}{\Rightarrow}t'$, or that there exist states $t_{i}$, $t'_{j}$ such that: $t\stackrel{\tau}{\rightarrow}t_{1}\stackrel{\tau}{\rightarrow}...\stackrel{\tau}{\rightarrow}t_{k}\stackrel{a}{\rightarrow}t'_{1}\stackrel{\tau}{\rightarrow}...\stackrel{\tau}{\rightarrow}t'_{m}\stackrel{\tau}{\rightarrow}t'$. Let $u_{i}\approx t_{i}$,...,$u'_{j}\approx t'_{j}$,.... It follows that $u\stackrel{\tau}{\Rightarrow}u_{1}\stackrel{\tau}{\Rightarrow}...\stackrel{\tau}{\Rightarrow}u_{k}\stackrel{a}{\Rightarrow}u'_{1}\stackrel{\tau}{\Rightarrow}...\stackrel{\tau}{\Rightarrow}u'$, or that there exist states $u''$, $u'''$ in U such that $u\left(\stackrel{\tau}{\rightarrow}\right)^{*}u_{k}\left(\stackrel{\tau}{\rightarrow}\right)^{*}u''\stackrel{a}{\rightarrow}u'''\left(\stackrel{\tau}{\rightarrow}\right)^{*}u'_{1}\left(\stackrel{\tau}{\rightarrow}\right)^{*}u'$ which means $u\stackrel{a}{\Rightarrow}u'$. Since $U^{*}$ is a saturation of $U$, it follows that $u\stackrel{a}{\rightarrow}u'\in U^{*}$.
Similarly, for every transition $u\stackrel{a}{\rightarrow}u'\in U^{*}$, there is a transition in $t\stackrel{a}{\rightarrow}t'\in T^{*}$, such that $u\approx t$, $u'\approx t'$.
Therefore, the weak bisimulation relation $\approx$ for $T$ and $U$ is a strong bisimulation relation for $T^{*}$ and $U^{*}$.

----- The algorithm for saturation

The set of triples ${R}$ can be partitioned in ${2n}$ sets with: \\
\ \ \ \ ${T_{\tau p}=\left\{\left(p,\tau,q\right)| \left(p,\tau,q\right)\in T\right\}}$, and\\
\ \ \ \ ${T_{ap}=\left\{\left(p,\tau,q\right)|\ a\neq\tau\wedge\left(p,\tau,q\right)\in T\right\}}$ for every ${p\in S}$.

By the definition of ${T_{\tau p}}$ and ${T_{ap}}$ it can be seen that ${\bigcup_{p\in S}\left(T_{\tau p}\cup T_{ap}\right)=T}$, and also, their pairwise intersection is empty. 

The family of sets ${T^{*}_{\tau p}}$ can be iteratively constructed with:\\
${T^{0}_{\tau p}=T_{\tau p}\cup\left\{\left(p,\tau,p\right)\right\}}$,\\
${T^{i}_{\tau p}=T^{i-1}_{\tau p}\cup\left\{\left(p,\tau,r\right)|\left(\exists q\in S\right)\left(p,\tau,q\right)\in T^{i-1}_{\tau p}\wedge\left(q,\tau,r\right)\in T_{\tau q}\right\}}$ and \\
${T^{*}_{\tau p}=T^{n}_{\tau p}}$

(Note: ${\left|T^{*}_{\tau p}\right|\leq\left|S\right|=n}$; when for some ${k<n}$, ${T^{k}_{\tau p}=T^{k+1}_{\tau p}}$, then ${T^{*}_{\tau p}=T^{k}_{\tau p}=T^{n}_{\tau p}}$)

With this step a reflexive, transitive closure was constructed: ${T^{*}_{\tau}=\bigcup_{p\in A}T^{*}_{\tau p}=\left\{\left(p,\tau,q\right)|p\left(\stackrel{\tau}{\rightarrow}\right)^{*}q\right\}}$.

\begin{figure}[!ht]
\centering
\includegraphics[width=4.5in]{saturation}
\caption{Reflexive, transitive closure of $\tau$. The original graph is depicted with red lines}
\label{fig:saturation}
\end{figure}

The next step is to construct $T'_{ap}=\bigcup_{p\in A}T'_{ap}=\left\{\left(p,a,q\right)|\left(\exists q'\in S\right)\left(p,a,q'\right)\in T\wedge q'\left(\stackrel{\tau}{\rightarrow}\right)^{*}q\right\}$:\\
$T^{0}_{ap}=T_{ap'}$,\\
$T^{i}_{ap}=T^{i-1}_{ap}\cup \left\{\left(p,a,r\right)|\left(\exists q\in S\right)\left(p,a,q\right)\in T^{i-1}_{ap}\wedge \left(q,\tau,r\right)\in T^{i-1}_{\tau q}\right\}$ and \\
$T'_{ap}=T^{n|Act|}_{ap}$

(Note: $|T'_{ap}|\leq |S||Act|=n|Act|$, when for some $k<n|Act|$, $T^{k}_{ap}=T^{k+1}_{ap}$, then $T'_{ap}=T^{k}_{ap}=T^{n|Act|}_{ap}$)

For the third step, $T'=\bigcup_{p\in S}\left(T^{*}_{\tau p}\cup T'_{ap}\right)$ needs to be partitioned again, defined by the destination in the triple:\\
$T_{\tau q}=\left\{\left(p,\tau,q\right)|\left(p,\tau,q\right)\in T'\right\}$ and $T_{bq}=\left\{\left(p,a,q\right)|a\neq\tau\wedge\left(p,a,q\right)\in T'\right\}$ for every $p\in S$,\\
and then construct:\\
$T^{0}_{bq}=T_{bq}$,\\
$T^{i}_{bq}=T^{i-1}_{bq}\cup\left\{\left(p',a,q\right)|\left(\exists p\in S\right)\left(p,a,q\right)\in T^{i-1}_{bq}\wedge\left(p',\tau,p\right)\in T^{*}_{\tau p}\right\}$ and\\
$T^{*}_{bq}=T^{n|Act|}_{bq}$

Finally the saturated LTS is:\\
$T^{*}=\bigcup_{p\in S}\left(T^{*}_{\tau p}\cup T^{*}_{bp}\right)=\left(\stackrel{\tau}{\rightarrow}\right)^{*}\cup\left{\left(p,a,q\right)|a\neq\tau\wedge\left(\exists p',q'\in A\right)p\left(\stackrel{\tau}{\rightarrow}\right)^{*}p'\stackrel{a}{\rightarrow}q'\left(\stackrel{\tau}{\rightarrow}\right)^{*}q\right}$

\begin{figure}[!ht]
\centering
\includegraphics[width=4.5in]{saturation2}
\caption{Example saturated LTS}
\label{fig:saturation2}
\end{figure}

    
    

 
%\section{Alternating Bit Protocol - modelling, specification and testing}
% no \IEEEPARstart
The alternating bit protocol is a simple yet effective protocol (usually used as a test case), designed to ensure reliable communication through unreliable transmission mediums, and it�s used for managing the retransmission of lost messages \cite{ReactiveSystems3}\cite{Kulick}.

The representation of Alternating Bit Protocol is shown bellow, and it consists of $Sender$ $S$, $Receiver$ $R$ and two channels $Transport$ $T$ and $Acknowledge$ $A$. In the following text Alternating Bit Protocol is abbreviated as ABP.

\begin{figure}[!ht]
\centering
\includegraphics[width=4.5in]{abp}
\caption{Alternating bit protocol}
\label{fig:abp}
\end{figure}

All of the transitions in the ABP are internal synchronization and the only visible transitions are deliver and accept, which can occur only sequentially. 

Here is the specification of ABP:\\
$ABP=\overline{deliver}.accept.ABP$

Messages are sent from a sender $S$ to a receiver $R$. Channel from $S$ to $R$ is initialized and there are no messages in transit. There is no direct communication between the sender $S$ and the receiver $R$, and all messages must travel trough the medium (transport and acknowledge channel). The ABP works like this:
\begin{enumerate}
	\item Each message sent by $S$ contains the protocol bit, 0 or 1.\\
	      Here is the implementation:\\
	      $Imp=\left(S|T|R|A\right)\backslash\left{send0,send1,receive0,receive1,reply0,reply1,ack0,ack1\right}$
	\item When a sender $S$ sends a message, it sends it repeatedly (with its corresponding bit) until receiving an acknowledgment ($ack0$ or $ack1$) from a receiver $R$ that contains the same protocol bit as the message being sent.\\
	      $S=\overline{send0}.S+ack0.accept.S_{1}+ack1.S$\\
	      $S_{1}=\overline{send1}.S_{1}+ack1.accept.S+ack0.S_{1}$\\
	      The transport channel transmits the message to the receiver, but it may lose the message (lossy channel) or transmit it several times (chatty channel).
	      $T=send0.\left(T+T_{1}\right)+send1.left(T+T_{2}\right)$\\
	      $T_{1}=\overline{receive0}.\left(T+T_{1}\right)$\\
	      $T_{2}=\overline{receive1}.\left(T+T_{2}\right)$
  \item When $R$ receives a message, it sends a reply to $S$ that includes the protocol bit of the message received. When a message is received for the first time, the receiver delivers it for processing, while subsequent messages with the same bit are simply acknowledged.
        $R=receive0.\overline{deliver}.R_{1}+\overline{reply1}.R+receive1.R$\\
        $R_{1}=receive1.\overline{deliver}.R+\overline{reply0}.R_{1}+receive0.R_{1}$\\
        Again the acknowledgement channel sends the ack to sender, and it also can acknowledge it several times or lose it on the way to the sender.\\
        $A=reply0.\left(A+A_{1}\right)+reply1.\left(A+A_{2}\right)$\\
        $A_{1}=\overline{ack0}.\left(A+A_{1}\right)$\\
        $A_{2}=\overline{ack1}.\left(A+A_{2}\right)$
  \item When $S$ receives an acknowledgment containing the same bit as the message it is currently transmitting, it stops transmitting that message, flips the protocol bit, and repeats the protocol for the next message.\cite{Kulick}\cite{ProcessAlgebraParallel}
\end{enumerate}
 
%\section{Peterson and Hamilton algorithm - Modeling,  Specification and Testing}
% no \IEEEPARstart
The algorithm

var flag: array [0..1] of boolean; 
turn: 0..1; 
%flag[k] means that process[k] is interested in the critical section 
flag[0] := FALSE; 
flag[1] := FALSE; 
turn := random(0..1) 

%After initialization, each process, which is called process i in the code, runs this code: 

repeat 
flag[i] := TRUE; 
turn := j; 
while (flag[j] and turn=j) do no-op; 
%CRITICAL SECTION 
flag[i] := FALSE; 
%REMAINDER SECTION 
until FALSE;

There are two different processes which want to enter at critical section. 
This means that the processes are fighting for one resource (ex. Say one variable or data structure). 
flag[i]= true means that process I wants to enter the critical section and the turn=i means that process i is next to enter the critical section.
 At first the shared variables flag[0] and flag[1] are initialized to false because neither process is yet interested in the critical section. 
 The shared variable turn is set to either 0 or 1 randomly (or it can always be set to say 0). 
 If the process can’t enter the critical section it waits in while loop.

Legend:

turn € {0, 1}
flag0, flag1 € {true, false} = {t, f}

w – write
r – read
enter – enter the critical section
exit – exit the critical section

Initialization:

turn=1
flag1 = flag2 = f

CCS Specification:

Peterson = (P1 | P2 | flag1f | flag2f | turn1) \ L

P1 = flag1wt.turnw2.P11
P11 = flag2rf.P12 + flag2rt.(turnr2. τ.P11 + turnr1.P12)
P12 = enter1.exit1.flag1wf.P1

P2 = flag2wt.turnw1.P21
P21 = flag1rf.P22 + flag1rt.(turnr1. τ.P21 + turnr2.P22)
P22 = enter2.exit2.flag2wf.P2

FLAG1f = flag1rf.flag1f + flag1wf.flag1f + flag1wt.flag1t
FLAG1t = flag1rt.flag1t + flag1wt.flag1t + flag1wf.flag1f

FLAG2f = flag2rf.flag2f + flag2wf.flag2f + flag2wt.flag2t
FLAG2t = flag2rt.flag2t + flag2wt.flag2t + flag2wf.flag2f

TURN1 = turnr1.turn1 + turnw1.turn1 + turnw2.turn2
TURN2 = turnr2.turn2 + turnw2.turn2 + turnw1.turn1

L = { flag1wt, flag1rt, turnw2,… union of the access sorts (r,w) of the
variables }

\begin{figure}[h!]
\centering
\includegraphics[width=1.0in]{model1}
% where an .eps filename suffix will be assumed under latex, 
% and a .pdf suffix will be assumed for pdflatex; or what has been declared
% via \DeclareGraphicsExtensions.
\caption{Model 1}
\label{fig:model1}
\end{figure}

\begin{figure}[h!]
\centering
\includegraphics[width=3.5in]{model2}
% where an .eps filename suffix will be assumed under latex, 
% and a .pdf suffix will be assumed for pdflatex; or what has been declared
% via \DeclareGraphicsExtensions.
\caption{Model 2}
\label{fig:model2}
\end{figure}

The mutual exclusion requirement is assured. Suppose instead that both processes are in their critical section. 
Only one can have the turn, so the other must have reached the while test before the process with the turn set its flag. 
But after setting its flag, the other process had to give away the turn. 
Contradicting our assumption, the process at the while test has already changed the turn and will not change it again.
	The progress requirement is assured. 
	Again, the turn variable is only considered when both processes are using, or trying to use, the resource.
	Deadlock is not possible. One of the processes must have the turn if both processes are testing the while condition. 
	That process will proceed.
	Finally, bounded waiting is assured. When a process that has exited the CS reenters, it will give away the turn. 
	If the other process is already waiting, it will be the next to proceed.

Test and set algorithm

repeat
while Test-and-Set(lock) do no-op;
critical section 
lock := false;
remainder section 
until false 

Test-and-Set(target) 
result := target;
target := true; 
return result 

\begin{tabular}{ | l | l | }
  \hline                       
  Process 1	& Process 2 \\ \hline 
  Wants to set target to true & \\
  Target is changed to true & \\
  Result comes back false so no busy waiting &\\
  & Wants to set target to true \\
  & It receives the result true \\
  & Busy waits as long as Process 1 is in critical section \\
  Leaves critical section & \\
  Set target to false & \\
  \hline  
\end{tabular}
 
\section{Conclusions and Future Work}

The conclusion goes here.

TODO: Write conclusion, include part for future work (future development of the tool) as well. 
% use section* for acknowledgement
\section*{Acknowledgment}
We would like to thank d-r Jasen Markovski for introducing us to the theory of formal methods and its application, being our mentor and guiding us all throughout the project with his suggestions and constructive feedback. 
% references section

% can use a bibliography generated by BibTeX as a .bbl file
% BibTeX documentation can be easily obtained at:
% http://www.ctan.org/tex-archive/biblio/bibtex/contrib/doc/
% The IEEEtran BibTeX style support page is at:
% http://www.michaelshell.org/tex/ieeetran/bibtex/
%\bibliographystyle{IEEEtran}
% argument is your BibTeX string definitions and bibliography database(s)
%\bibliography{IEEEabrv,../bib/paper}
%
% <OR> manually copy in the resultant .bbl file
% set second argument of \begin to the number of references
% (used to reserve space for the reference number labels box)
\begin{thebibliography}{1}

\bibitem[AILS07]{ReactiveSystems}
Luca Aceto, Anna Ingolfsdottir, Kim G. Larsen and Jiri Srba,
\emph{Reactive Systems - Modeling, Specification and Verification},
\hskip 1em plus 0.5em minus 0.4em\relax Cambridge University Press, 2007

\bibitem[Par07]{ANTLRRef}
Terence Parr, \emph{The Definitive ANTLR Reference -
 Building Domain-Specific Languages}\emph, The Pragmatic Bookshelf, 2007

\bibitem[NW]{NiklausWirth}
Niklaus Wirth
\newline \url{http://en.wikipedia.org/wiki/Niklaus_Wirth}

\bibitem[BK08]{ModelChecking}
Christel Baier, Joost-Pieter Katoen, 
\emph{Principles of model checking},
\hskip 1em plus 0.5em minus 0.4em\relax The MIT Press, pages 456-463, 2008

\bibitem[AILS07a]{ReactiveSystems1}
Luca Aceto, Anna Ingolfsdottir, Kim G. Larsen and Jiri Srba,
\emph{Reactive Systems - Modeling, Specification and Verification},
\hskip 1em plus 0.5em minus 0.4em\relax Cambridge University Press, pages 85-88, 2007

\bibitem[AILS07b]{ReactiveSystems2}
Luca Aceto, Anna Ingolfsdottir, Kim G. Larsen and Jiri Srba,
\emph{Reactive Systems - Modeling, Specification and Verification},
\hskip 1em plus 0.5em minus 0.4em\relax Cambridge University Press, pages 78-85, 2007

\bibitem[Fer89]{Fernandez}
J.-C. Fernandez, \emph{An Implementation of an Efficient Algorithm for Bisimulation
Equivalence}, Science of Computer Programming, vol. 13, pages 219-236, 1989/1990

\bibitem[PT87]{PaigeTarjan}
R. Paige and R. Tarjan, \emph{Three partition refinement algorithms}\emph, 
SIAM J. Comput. 16 (6), 1987

\bibitem[BPS01]{HandbookProcessAlgebra}
J.A. Bergstra, A. Ponse and S.A. Smolka, \emph{Handbook of process algebra}\emph, 
Elsevier Science B.V., pages 9-13, 2001


\end{thebibliography} 

\appendix
\section*{Appendix A: Comparison and Analysis of the bisimulation equivalence algorithms}
\label{appendixA}

There is no official set of benchmarks for testing algorithms for computing bisimulation equivalence \cite{PiazzaPolicriti}. And it is also very difficult to randomly generate labeled transition systems suitable for proper testing of bisimulation equivalence. Therefore, we used the academical examples from mCRL2 as experimental test models. 

The experiments were conducted on a portable computer with the following specifications: 
\begin{enumerate}
	\item CPU: Intel Pentium P6100 2.0 GHz
	\item Memory: 3 GB
	\item OS: Windows 2007 Premium
\end{enumerate}

Each of the experiments carried out on our Java implementations of the two algorithms for computing bisimulation equivalence consisted of 10 repeated runs of each of the algorithms on each of the aldebaran test files and measuring the average running time in miliseconds. 
The results from these experiments are presented in Table~\ref{table2}. The table includes the running times in miliseconds and the absolute error in miliseconds for both of the algorithms, the number of pairs of bisimilar states obtained with the naive algorithm (excluding the reflexive and symmetric pairs) and the number of classes of bisimulation equivalence obtained with the algorithm due to Fernandez (excluding the one-element classes), as well as the ratio of the running time of the naive algorithm with respect to the running time of Fernandez's algorithm.

\begin{table}
\begin{tabular}{| l | l | l | l | l | l | l | p{1.3cm}  | p{1.3cm} | l | }

	\hline 
	\multicolumn{3}{|c} { }
	& \multicolumn{2}{|c|}{Naive}
	& \multicolumn{2}{|c|}{Fernandez}
	& \multicolumn{2}{|c|} { }
	& Ratio
	\\ \hline  
  \hline                       
	aut &
	states &
	transitions &
	$t(ms)$ &
	$\Delta t(ms)$ &
	$t(ms)$ &
	$\Delta t(ms)$ &
	bisimilar pairs &
	bisimilar classes &
	$t_{n}/t_{f}$
	\\ \hline
	
	dining3 &
	93 &
	431 &
	16.8 &
	0.8 &
	250 &
	2 &
	1 &
	2 &     
	0.067
	\\ \hline
	
	abpbw &
    97 &
    122 &
    287.9 &
    0.94 &
    173.3 &
    5.22 &
    32 &
    27 &
    1.661   
    \\ \hline
	
	abp &
    74 &
    92 &
    162.2 &
    1.44 &
    69.6 &
    3.92 &
    6 &
    6 &
    2.33   
    \\ \hline
	
	scheduler &
	13 &
	19 &
	17.4 &
	1.18 &
	4.4 &
	0.96 &
	1 &
	1 &
	3.95   
	\\ \hline
	
	mpsu &
	52 &
	150 &
	215.4 &
	1.16 &
	27.5 &
	0.5 &
	4 &
	4 &
	7.83   
	\\ \hline
  
    trains &
    32 &
    52 &
    64.5 &
    7.3 &
    7.6 &
    1.76 &
    6 &
    6 &
    8.48   
    \\ \hline
	
    par &
    94 &
    121 &
    5909.4 &
    22.68 &
    28.5 &
    0.7 &
    170 &
    27 &
    207.34   
    \\ \hline  
  
    leader &
    392 &
    1128 &
    / &
    / &
    841.5 &
    41.5 &
    / &
    20 &
    /   
    \\ \hline
  
    tree &
    1025 &
    1024 &
    / &
    / &
    1858.1 &
    55.16 &
    / &
    8 &     
    / \\ \hline 
  
    cabp &
    672 &
    2352 &
    / &
    / &
    16184.3 &
    198.94 &
    / &
    90 &     
    /
   \\ \hline
  
\end{tabular}
\caption{Results of the comperisons}
\label{table2}
\end{table}
The results from the experiments carried out are presented graphically in Fig.~\ref{fig:naiveAnalysis} and Fig.~ \ref{fig:fernandezAnalysis} for the naive algorithm and the algorithm of Fernandez, respectively. 

\begin{figure}[h]
\centering
\includegraphics[width=5.0in]{naive}
\caption{Analysis of the running time of the naive bisimulation algorithm}
\label{fig:naiveAnalysis}
\end{figure}

\begin{figure}[h]
\centering
\includegraphics[width=5.0in]{fernandez}
\caption{Analysis of the running time of the bisimulation algorithm due to Fernandez}
\label{fig:fernandezAnalysis}
\end{figure}

It can be seen that for all test models, with the exception of the first one, the running time of the naive algorithm is proportional to the number of states, the number of transitions and the number of resulting bisimilar pairs. However, we need to note that this is not a general conclusion and that in general the running time of the naive algorithm for computing bisimulation equivalence depends on the nature of the monotonic function used by this algorithm, which is strongly related with the form of the labeled transition system. As an example, dining3 is a labeled transition system that has big number of transitions, however in this test model the algorithm determines that the monotony condition is not fulfilled in the second development of the monotonic function and therefore the algorithm stops there.

Similar conclusions can be drawn for the algorithm due to Fernandez. Its running time is also proportional to the number of transitions, number of states and the number of bisimilar classes. 

The comparison of running times of the two algorithms obtained with the experiment is shown in Fig.~\ref{fig:comparison1}. As it can be clearly seen the algorithm of Fernandez is few times faster than the naive algorithm. The biggest difference can be noticed for the example par.aut where the ratio of the running times is 207.34. An exception is only the example dining3.aut for which the naive algorithm performs faster due to the reasons described earlier. The ration of the running times in this case is 0.0067. Fig.~ \ref{fig:comparison2} shows the ratios of the running times of the two algorithms for each of the examples.

\begin{figure}[h]
\centering
\includegraphics[width=5.0in]{compare1}
\caption{Comparison of the running time of the two bisimulation algorithms}
\label{fig:comparison1}
\end{figure}

\begin{figure}[h]
\centering
\includegraphics[width=5.0in]{compare2}
\caption{Ratio of the running times of the two bisimulation algorithms}
\label{fig:comparison2}
\end{figure}
\section*{Appendix B: CCS grammar}
\label{appendixB}

One of the few standard ways of showing a grammar is a syntax diagram. On Fig. \ref{fig:CCSGrammar} we show the syntax diagram
for the grammar that recognizes CCS expressions and is used in our tool.

\begin{figure}
\centering
\includegraphics[width=5.0in]{CCSGrammar}
\caption{Syntax diagram for the CCS grammar}
\label{fig:CCSGrammar}
\end{figure}
\section*{Appendix C: HML grammar}
\label{appendixC}

The grammar that recognizes HML expressions and is used in our tool is given below.

\begin{enumerate}
\item $HML \Rightarrow$ Term Expression
\item Expression $\Rightarrow$ $\hspace{1 mm} AND \hspace{1 mm} $ Term Expression
\item Expression $\Rightarrow$ $\hspace{1 mm} OR \hspace{1 mm} $ Term Expression
\item Expression $\Rightarrow$ $\hspace{1 mm} Uw \hspace{1 mm} $ Term Expression
\item Expression $\Rightarrow$ $\hspace{1 mm} Us \hspace{1 mm} $ Term Expression
\item Expression $\Rightarrow$ $\lambda$
\item Term $\Rightarrow$ $NOT$ Term
\item Term $\Rightarrow$ [ Actions ] Term
\item Term $\Rightarrow$ $\langle$ Actions $\rangle$ Term
\item Term $\Rightarrow$ $TT$
\item Term $\Rightarrow$ $FF$
\item Term $\Rightarrow$ ( $HML$ )
\item Actions $\Rightarrow$ Action
\item Actions $\Rightarrow$ { Action ActionsList }
\item ActionsList $\Rightarrow$ Action ActionsList
\item ActionsList $\Rightarrow$ $\lambda$
\item Action $\Rightarrow$ Literal Name
\item Name $\Rightarrow$ Literal
\item Name $\Rightarrow$ Number
\item Name $\Rightarrow$ $\lambda$
\item Literal $\Rightarrow \hspace{1 mm}  a  \hspace{1 mm} \vert \hspace{1 mm}  b  \hspace{1 mm}  \vert c \hspace{1 mm}  \vert \hspace{1 mm} d \hspace{1 mm} \vert \hspace{1 mm} e \hspace{1 mm} \vert \hspace{1 mm} f \hspace{1 mm} \vert \hspace{1 mm} g \hspace{1 mm} \vert \hspace{1 mm} h \hspace{1 mm} \vert \hspace{1 mm} i \hspace{1 mm} \vert \hspace{1 mm} j \hspace{1 mm} \vert \hspace{1 mm} k \hspace{1 mm} \vert \hspace{1 mm} l \hspace{1 mm} \vert \hspace{1 mm} m \hspace{1 mm} \vert \hspace{1 mm} n \hspace{1 mm} \vert \hspace{1 mm} o \hspace{1 mm} \vert \hspace{1 mm} p \hspace{1 mm} \vert \hspace{1 mm} q \hspace{1 mm} \vert \hspace{1 mm} r \hspace{1 mm} \vert \hspace{1 mm} s \hspace{1 mm} \vert \hspace{1 mm} t \hspace{1 mm} \vert \hspace{1 mm} u \hspace{1 mm} \vert \hspace{1 mm} v \hspace{1 mm} \vert \hspace{1 mm} w \hspace{1 mm} \vert \hspace{1 mm} x \hspace{1 mm} \vert \hspace{1 mm} y \hspace{1 mm} \vert \hspace{1 mm} z$
\item Number $\Rightarrow \hspace{1 mm} 0 \hspace{1 mm} \vert \hspace{1 mm} 1 \hspace{1 mm} \vert \hspace{1 mm} 2 \hspace{1 mm} \vert \hspace{1 mm} 3 \hspace{1 mm} \vert \hspace{1 mm} 4 \hspace{1 mm} \vert \hspace{1 mm} 5 \hspace{1 mm} \vert \hspace{1 mm} 6 \hspace{1 mm} \vert \hspace{1 mm} 7 \hspace{1 mm} \vert \hspace{1 mm} 8 \hspace{1 mm} \vert \hspace{1 mm} 9$
\end{enumerate}



\section*{Appendix D: Tool Usage}
\label{appendixD}

In this section we demonstrate how to use TMACS to show that the specification and 
implementation of the alternating bit protocol, as described in Section~\ref{sec:application} are weakly bisimiliar. 
As a first step we always need to generate the labeled transition systems for both the specification and the 
implementation of the system that we are trying to model. That can be done in the tab "CCS to LTS" in the tool. 
As shown in Fig.~\ref{fig:abptoolusage1} in the upper text area we need to write down the CCS 
expressions that describe the system or load them from a file. One constraint here
is that a general CCS expression that describes the whole system has to be put on
the first line. The algorithm for generating the labeled graph performs the evaluation and the 
construction of the graph starting from the first line. When we have our CCS expressions
put in place we have to click "Generate LTS". This action will make the application
parse the expression and if no errors are found it will start generating the labeled graph.
The results from the generated graph will be presented in Aldebaran format in the lower text area. 
The users can save the results in a file or can click "View LTS Graph" 
which causes the tool to display a computer generated image of the labeled graph.

\begin{figure}[!ht]
\centering
\includegraphics[width=5in]{ABPToolUsage1}
\caption{TMACS usage: ABP modeling and generating the respective labeled transition system in Aldebaran format}
\label{fig:abptoolusage1}
\end{figure}

Reduction of the state space of a labeled transition system can be done in the "LTS minimization" tab. The minimization is pretty 
straigthforward process. As shown in Fig.~\ref{fig:abptoolusage2} and Fig.~\ref{fig:abptoolusage3}
the labeled transtion system is loaded from an Aldebaran file, a method of calculation is chosen and the user has to click the button "Calculate" in order to perform calculation.

\begin{figure}[!ht]
\centering
\includegraphics[width=5in]{ABPToolUsage2}
\caption{TMACS usage: ABP minimization, using weak bisimilarity and the naive algorithm for computing strong bisimulation equivalence}
\label{fig:abptoolusage2}
\end{figure}

\begin{figure}[h]
\centering
\includegraphics[width=5in]{ABPToolUsage3}
\caption{TMACS usage: ABP minimization, using strong bisimilarity and the algorithm of Fernandez for computing strong bisimulation equivalence}
\label{fig:abptoolusage3}
\end{figure}

Checking for labeled transition system bisimilarity is performed in the "LTS comparison" tab. This tab is 
similar to the one for the minimization. As shown in Fig.~\ref{fig:abptoolusage4} and
Fig.~\ref{fig:abptoolusage5} two labeled transition systems have to be loaded from Aldebaran files. In order to perform comparison of the two labeled transition systems with respect to strong or weak bisimulation equivalence, the user has to choose a method for calculation and then click the button 
"Calculate". The results will be displayed on the right side of the button and they
give information whether the labeled transition systems are strongly/weakly bisimilar and how long the calculation lasted.

\begin{figure}[!ht]
\centering
\includegraphics[width=5in]{ABPToolUsage4}
\caption{TMACS usage: ABP comparison, using weak bisimilarity and the algorithm of Fernandez for computing strong bisimulation equivalence}
\label{fig:abptoolusage4}
\end{figure}

\begin{figure}[!ht]
\centering
\includegraphics[width=5in]{ABPToolUsage5}
\caption{TMACS usage: ABP comparison, using strong bisimilarity and the naive algorithm for computing strong bisimulation equivalence}
\label{fig:abptoolusage5}
\end{figure}

TMACS also gives possibility to check whether certain Hennessy-Milner logic expression is valid for a given labeled transition system in Aldebaran format. That is done via the "Hennessy-Milner" tab. The labeled transition system is loaded from an Aldebaran file and then the user needs to enter Hennessy-Milner logic recursive formula which uses $U^{w}$ and/or $U^{s}$ operator. By clicking on the "Calculate" button, a true/false answer is given meaning that the Hennessy-Milner expression is valid or not for the input labeled transition system. This process is shown on Fig.~\ref{fig:hml1} and Fig.~\ref{fig:hml2}.

\begin{figure}[!ht]
\centering
\includegraphics[width=5in]{hml1}
\caption{TMACS usage: checking validity of a recursive Hennessy-Milner logic formula for an input labeled transition system}
\label{fig:hml1}
\end{figure}

\begin{figure}[!ht]
\centering
\includegraphics[width=5in]{hml2}
\caption{TMACS usage: checking validity of a recursive Hennessy-Milner logic formula for an input labeled transition system}
\label{fig:hml2}
\end{figure}

\end{document}
