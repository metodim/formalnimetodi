\section{Minimization and comparison of labeled transition systems}
\label{sec:bisimulation}

Bisimulation equivalence (bisimilarity) \cite{Park} is a binary relation between labeled transition systems which associates systems that can simulate each other's behaviour in a stepwise manner, enabling comparison of different transition systems \cite{ModelChecking}. An alternative perspective is to consider the bisimulation equivalence as a relation between states of a single labeled transition system. Using the quotient transition system under such a relation, smaller models of the labeled transition system can be obtained \cite{ModelChecking}.

%The bisimulation equivalence finds its extensive application in many areas of computer science such as concurrency theory, formal verification, set theory, etc. For instance, in formal verification minimization with respect to bisimulation equivalence is used to reduce the size of the state space to be analyzed. Also, bisimulation equivalence is of particular interest in model checking, in specific to check the equivalence of an implementation of a certain system with respect to its specification model.

TMACS implements both options: reducing the size of the state space of a given labeled transition system and checking the equivalence of two labeled transition systems, using two behavioral equivalence relations: strong bisimilarity and observational equivalence (weak bisimilarity).

\subsection{Minimization of a labeled transition system modulo strong bisimilarity}
The process of reducing the size of the state space of a labeled transition system $A=\left(S, Act, \rightarrow, s, F \right)$ in TMACS was implemented using an approach which consists of two steps:
\begin{enumerate}
\item Computing strong bisimulation equivalence (strong bisimilarity) for the labeled transition system;
\item Minimizing the labeled transition system to its canonical form using the strong bisimilarity obtained in the first step;
\end{enumerate}

Two different methods were used for computing the strong bisimulation equivalence: the so called naive method and a more efficient 
method due to Fernandez, both of which afterwards serve as minimization procedures.

The naive algorithm for computing strong bisimulation over finite labeled transition systems stems directly from the theory of partially ordered sets and lattices \cite{ReactiveSystems} underlying Tarski's classic fixed point theorem \cite{Tarsky}. It is based on an alternative definition of the notion of strong bisimilarity as the largest fixed point of the monotic function $\mathcal{F}$ defined as follows \cite{ReactiveSystems}:\\
\\
Let $\mathcal{R}$ is a binary relation over $\mathit{S}$, that is, an element of the set $2^{\mathit{S}\times \mathit{S}}$. Then $(p,q)\in \mathcal{F(R)}$ for all $p,q\in \mathit{S}$ iff:
	\begin{enumerate}
		\item $p\stackrel{a}{\rightarrow}p'$ implies $q\stackrel{a}{\rightarrow}q'$ for some $q'$ such that $(p',q')\in\mathcal{R}$;
		\item $q\stackrel{a}{\rightarrow}q'$ implies $p\stackrel{a}{\rightarrow}p'$ for some $p'$ such that $(p',q')\in\mathcal{R}$;
	\end{enumerate}

The algorithm that stems from this interpretation of the strong bisimulation equivalence has time complexity of $O(mn)$ for a labeled transition system with \emph{m} transitions and \emph{n} states. Its implementation in TMACS takes as an input a labeled transition system in Aldebaran format, and generates the corresponding labeled graph as a list of nodes representing the states which use the following data structure:
\begin{itemize}
	\item $S_p=\{(a, q)\hspace{1mm}|\hspace{1mm} p\stackrel{a}{\rightarrow}q, \hspace{1mm} p,q\in \mathit{S}, \hspace{1mm} a\in \mathit{Act}\}$ - set of pairs $(a, q)$ for state $p$ where $a$ is an outgoing action for $p$ and $q$ is a state
	reachable from $p$ with the action $a$
\end{itemize}
The algorithm then computes the strong bisimulation equivalence and outputs it as pairs of bisimilar states:
\begin{equation*}
	L = \left\{\left(p,q\right)|\hspace{1mm}\hspace{1mm} p\sim q, \hspace{1mm}p,q\in \mathit{S}\right\}
\end{equation*}

The algorithm due to Fernandez exploits the idea of the relationship between strong bisimulation equivalence 
and the relational coarsest partition problem \cite{KanellakisSmolka}\cite{PaigeTarjan}. Paige and Tarjan \cite{PaigeTarjan} 
proposed an algorithm that computes the relational coarsest partition problem in $O(m \log n)$ time and $O(m)$ space for a labeled transition system with $m$ transitions and $n$ states. Fernandez adapted the algorithm of Paige-Tarjan by considering a family of relations $\left(T_a\right)_{a\in \mathit{Act}}$ instead of one relation, with $T_a=\{(p,q)|p\stackrel{a}{\rightarrow}q\}$ as a transition relation for action $a\in \mathit{Act}$ \cite{Fernandez}. The adapted version has the same $O(m \log n)$ time complexity as the original one, major difference being that a refinement step is made with only one element of $Act$ in the original one. 

Our implementation of Fernandez's algorithm in TMACS takes a labeled transition system in Aldebaran format as an input, generates a labeled graph and then partitions the labeled graph into its coarsest blocks where each block represents a set of bisimilar states. Partition is a set of mutually exclusive blocks whose union constitutes the graph universe \cite{Fernandez}. To define graph states 
and transitions we used the following terminology represented by suitable data structures: 
\begin{itemize}
	\item $T_a[p]=\{q\}$ - an $a$-transition from state $p$ to state $q$
	\item $T_a{}^{-1}[q]=\{p\}$ - an inverse $a$-transition from state $q$ to state $p$
	\item $T_a{}^{-1}[B]=\cup \left\{T_a{}^{-1}[q],q\in B\right\}$ - inverse transition for block $B$ and action $a$
	\item $W$ - set of sets called splitters that are being used to split the partition
	\item infoB$(a, p)$ - info map for block $B$, state $p$ and action $a$
\end{itemize}

The algorithm of Fernandez outputs the strong bisimulation equivalence relation over $\mathit{Proc}$ as a partition $P=\left\{B_{1},...,B_{n}\right\}$ where $B_{i}$, $i=\overline{1,n}$, represent its equivalence classes:
\begin{equation*}
	P = \left\{B_{i}\hspace{1mm}|\hspace{1mm} p\approx q,\hspace{1mm} \forall p,q\in B_{i}, i=\overline{1,n} \right\}
\end{equation*}

Having computed the strong bisimulation equivalence, the next step in the reduction of the state space of the labeled transition system uses the bisimulation equivalence obtained in the first step in order to minimize the labeled graph. This reduction is implemented as follows:
\begin{enumerate}
	\item All states in a bisimilar equivalence class $B_{i}$ are merged into one single state $k=\bigcup p_{j}$, for $p_{j}\in B_{i}$;
	\item All incoming transitions $r \stackrel{a}{\rightarrow} p_{j}$, for $p_{j}\in B_{i}$, are replaced by transitions $r \stackrel{a}{\rightarrow} k$;
	\item All outgoing transitions $p_{j} \stackrel{a}{\rightarrow} t$, for $p_{j}\in B_{i}$, are replaced by transitions $k \stackrel{a}{\rightarrow} t$;
	\item The duplicate transitions are not taken into consideration.
\end{enumerate}
The procedure is repeated for all equivalence classes $B_{i}$, $i\in \overline{1,n}$.

This process of reduction with respect to strong bisimilarity is illustrated below in Fig.~\ref{fig:bisimGraph1}. The results obtained by applying both algorithms for computing strong bisimulation equivalence are given in Table~\ref{table1}. These results are then used as a basis for the reduction of the graph to its minimal form: all mutually bisimilar states are merged into a single state and their transitions are updated accordingly. 
\begin{figure}[h]
	\centering
	\includegraphics[width=3.0in]{bisimGraph1}
	\caption{Example of a labeled transition system graph and its minimial form modulo strong bisimilarity. The red dashed arrows depict the mapping of the states using the computed bisimulation equivalence. States 2 and 3 are merged into state 2 in the minimal graph, and states 4,5 and 6 are merged into state 3 in the minimal graph.}
	\label{fig:bisimGraph1}
\end{figure}

\begin{table}[h]
\begin{tabular}{| l | p{10.5cm}| }
  \hline                       
  Algorithm & Results \\ \hline
  Naive & (2, 3), (3, 2), (4, 5), 
(5, 4), (4, 6), (6, 4), (5, 6), (6, 5), (0, 0), (1, 1), (2, 2), (3, 3), (4, 4), (5, 5), (6, 6) \\ \hline
  Fernandez & \{0\}, \{1\}, \{2\}, \{3\}, \{4, 5, 6\} \\ \hline  
\end{tabular}
\\
\caption{Computing strong bisimularity for the example labeled graph from Fig.~\ref{fig:bisimGraph1}}
\label{table1}
\end{table}

\subsection{Minimization of a labeled transition system modulo weak bisimilarity}
The minimization of a labeled transition system modulo weak bisimilarity is reduced to the problem of minimization modulo strong bisimilarity, using a technique called saturation. Intuitevely, saturation amounts to first precomputing the weak transition relation and then constructing a new pair of finite processes whose original transitions are replaced by the weak transitions \cite{ReactiveSystems}. Once the algorithm for saturation is run and the original labeled transition system is saturated, the computation of weak bisimilarity amounts to computing strong bisimilarity over the saturated system.

The algorithm for saturation in TMACS was implemented as follows: 
\begin{enumerate}
\item For every ${p\in \mathit{S}}$, the set of all transitions $T$ of a labeled transition system with $m$ transitions and $n$ states is partitioned in ${2n}$ sets with: 
\begin{equation*}
	\begin{array}{lcl}
 		{T_{\mathit{\tau p}}=\left\{\left(p,\tau,q\right)| \left(p,\tau,q\right)\right\}}, \text{and}\\
    {T_{ap}=\left\{\left(p,a,q\right)| \left(p,a,q\right)\wedge a\neq\tau\right\}}
  \end{array}
\end{equation*} 
By the definition of ${T_{\mathit{\tau p}}}$ and ${T_{ap}}$, it follows that ${\bigcup_{p\in \mathit{S}}\left(T_{\mathit{\tau p}}\cup T_{ap}\right)=T}$, and also, that their pairwise intersection is empty. \\
The family of sets ${T^{'}_{\tau p}}$ is then iteratively constructed with:
\begin{equation*}
	\begin{array}{lcl}
		{T^{0}_{\mathit{\tau p}}=T_{\mathit{\tau p}}\cup\left\{\left(p,\tau,p\right)\right\}},\\
		{T^{i}_{\mathit{\tau p}}=T^{i-1}_{\mathit{\tau p}}\cup\left\{\left(p,\tau,q'\right)|\left(\exists q\in \mathit{S}\right)\left(p,\tau,q\right)\in T^{i-1}_{\mathit{\tau p}}\wedge\left(q,\tau,q'\right)\in T_{\mathit{\tau q}}\right\}}, \text{and} \\
		{T^{'}_{\mathit{\tau p}}=T^{n}_{\mathit{\tau p}}}
	\end{array}
\end{equation*}
(Note: ${\left|T^{'}_{\mathit{\tau p}}\right|\leq\left|\mathit{S}\right|=n}$; when for some ${k<n}$ it holds that ${T^{k}_{\mathit{\tau p}}=T^{k+1}_{\mathit{\tau p}}}$, then ${T^{'}_{\mathit{\tau p}}=T^{k}_{\mathit{\tau p}}=T^{n}_{\mathit{\tau p}}}$)\\

With this step a reflexive, transitive closure of $\tau$ is constructed:
\begin{equation*}
	{T^{*}_{\tau}=\bigcup_{p\in \mathit{S}}T^{'}_{\tau p}=\left\{\left(p,\tau,q\right)|\hspace{1mm}p\left(\stackrel{\tau}{\rightarrow}\right)^{*}q\right\}}
\end{equation*}

An example of a reflexive, transitive closure of $\tau$ as computed in this step, is shown on Fig.~\ref{fig:saturation}.\\

\begin{figure}[h]
\centering
\includegraphics[width=4.5in]{saturation}
\caption{Reflexive, transitive closure of $\tau$. The original graph is depicted with red lines.}
\label{fig:saturation}
\end{figure}

\item The next step is to construct
\begin{equation*}\label{eq:tap}
		T'_{s}=\bigcup_{p\in \mathit{S}}T'_{ap}=\left\{\left(p,a,q\right)|\left(\exists q'\in \mathit{S}\right)\left(p,a,q'\right)\in T\wedge q'\left(\stackrel{\tau}{\rightarrow}\right)^{*}q\right\}
\end{equation*}
as follows:
\begin{equation*}
	\begin{array}{lcl}
		T^{0}_{sp}=T_{sp},\\
		T^{i}_{sp}=T^{i-1}_{sp}\cup \left\{\left(p,a,q'\right)|\left(\exists q\in \mathit{S}\right)\left(p,a,q\right)\in T^{i-1}_{sp}\wedge \left(q,\tau,q'\right)\in T^{i-1}_{\tau q}\right\} \text{and} \\
		T'_{sp}=T^{n|\mathit{Act}|}_{sp}
	\end{array}
\end{equation*}

(Note: $|T'_{sp}|\leq |\mathit{S}||\mathit{Act}| = n|\mathit{Act}|$, and when for some $k < n|\mathit{Act}|$ it holds that $T^{k}_{sp}=T^{k+1}_{sp}$, then $T'_{sp}=T^{k}_{sp}=T^{n|\mathit{Act}|}_{sp}$)\\

\item For the third step, $T'=\bigcup_{p\in \mathit{S}}\left(T^{'}_{\mathit{\tau p}}\cup T'_{sp}\right)$ needs to be partitioned again, defined by the destination in the transition triple:
\begin{equation*}
	\begin{array}{lcl}
		T^{*}_{\mathit{\tau q}}=\left\{\left(p,\tau,q\right)|\left(p,\tau,q\right)\in T'\right\}, \text{and}\\
		T_{dq}=\left\{\left(p,a,q\right)|\hspace{1mm}\left(p,a,q\right)\in T' a\neq\tau\right\}				    
	\end{array}
\end{equation*}
for every $p\in \mathit{S}$, and then construct:
\begin{equation*}
	\begin{array}{lcl}
		T^{0}_{dq}=T_{dq},\\
		T^{i}_{dq}=T^{i-1}_{dq}\cup\left\{\left(p',a,q\right)|\left(\exists p\in \mathit{S}\right)\left(p,a,q\right)\in T^{i-1}_{dq}\wedge\left(p',\tau,p\right)\in T^{*}_{\tau p}\right\} \text{, and}\\
		T^{*}_{dq}=T^{n|\mathit{Act}|}_{dq}
	\end{array}
\end{equation*}
\end{enumerate}

Finally the saturated labeled transition system now is:
\begin{align*}
	T^{*} &=\bigcup_{p\in \mathit{S}}\left(T^{*}_{\mathit{\tau p}}\cup T^{*}_{dp}\right)=\\
	&=\left(\stackrel{\tau}{\rightarrow}\right)^{*}\cup\left\{\left(p,a,q\right)|\hspace{1mm}a\neq\tau\wedge\left(\exists p',q'\in S\right)\hspace{1mm} p\left(\stackrel{\tau}{\rightarrow}\right)^{*}p'\stackrel{a}{\rightarrow}q'\left(\stackrel{\tau}{\rightarrow}\right)^{*}q\right\}
\end{align*}
An illustration of a saturated labeled transition system is given in Fig.~\ref{fig:saturation2} for the original labeled transition system Fig.~\ref{fig:saturation3}:

\begin{figure}[h]
\centering
\includegraphics[width=3.5in]{saturation3}
\caption{Example of a labeled graph before saturation}
\label{fig:saturation3}
\end{figure}
\begin{figure}[h]
\centering
\includegraphics[width=4.5in]{saturation2}
\caption{The labeled graph from Fig.~\ref{fig:saturation3} after saturation}
\label{fig:saturation2}
\end{figure}

Having computed the observational equivalence (weak bisimilarity) of the original labeled transition system, the process of its minimization is the same as the process for minimization modulo strong bisimilarity applied on the saturated labeled transition system.

\subsection{Comparison of two labeled transition systems modulo strong bisimilarity}
The idea for the implementation of the equivalence checking of two labeled transition systems modulo strong bisimilarity was based on the following fact: Two labelled transition systems are (strongly) bisimilar iff their initial states are bisimilar \cite{ModellingAndAnalysis}. That means that in order to check whether two labeled transition systems are bisimilar it is enough to check whether their initial states are bisimilar. This can be done using the following approach:
\begin{enumerate}
	\item The two labeled transition systems are merged into a single transition system
	\item An algorithm for computing the strong bisimilarity is applied to the merged system
	\item A check is performed to see if the initial states belong to the same bisimulation equivalence class
\end{enumerate}

\subsection{Comparison of two labeled transition systems modulo weak bisimilarity}
The comparison of two labeled transition systems modulo weak bisimilariy amounts to checking strong bisimilarity over the saturated labeled transition systems \cite{ReactiveSystems}. In another words, two labeled transition systems are weakly bisimilar iff their saturated labeled transition systems are strongly bisimilar. Following this fact, we implemented the comparison of two labeled transition systems modulo weak bisimilarity by applying the saturation algorithm over the original labeled graphs in order to obtain their saturated labeled graphs, after which the process of comparison of the saturated labeled transition systems modulo strong bisimilarity was applied as described above.