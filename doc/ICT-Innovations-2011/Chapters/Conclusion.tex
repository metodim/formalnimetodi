\section{Conclusions and Future Work}
\label{sec:conclusion}

In this paper we presented TMACS, a tool for modeling, manipulation and analysis of concurrent systems. It includes recognizing CCS expressions, building labeled transition system graphs and checking equivalence between two labeled graphs with respect to strong and weak bisimilarity. The tool is simple, but yet functional enough to perform specification and verification of concurrent systems described as CCS expressions and/or labeled transition systems. TMACS is already successfully used for the needs of the Official Gazette of the Republic of Macedonia, a fact which indicates that the tool can be quite useful for practical as well as experimental applications.
%We have successfully validated and tested the tool with large number of examples comparing the results obtained with mCRL2 and CWB, and we also presented few examples showing the tool usage. 

%TMACS is a tool that has lots of potential for further development. 
Future work can include optimization of the tool's response times for very large labeled transition system graphs, as well as implementation of prunung strategy for infinite labeled transition system graphs. It would be also usefull if the labeled graph is better visualized where every SOS rule is shown, with its source and sink nodes showing the CCS expression for each of the nodes. 
%Such a visualization can be very helpfull for understanding the semantics of the CCS language for students that are starting to study it. 
Furthermore, the minimization and comparison functionality for labeled transition systems which at the moment includes only strong and weak bisimilarity can be extended to include other behavioural equivalences and preorders as well.
Another direction for future development can be to extend the tool to support not only CCS expressions and process semantics in general, but also other formal verification approaches as well. 