\section{CCS parsing and labeled transition system generation}
\label{sec:parsing}

Given a set of action names, the set of CCS processes is defined by the following BNF grammar:

\begin{equation}\label{eq:ccs_bnf}
P ::= \emptyset \hspace{1 mm} | \hspace{1 mm} a.P_{1} \hspace{1 mm} | \hspace{1 mm} A \hspace{1 mm} | \hspace{1 mm}P_{1}+P_{2} \hspace{1 mm} |
\hspace{1 mm} P_{1} | P_{2} \hspace{1 mm} | \hspace{1 mm} P_{1}[b/a] \hspace{1 mm} | \hspace{1 mm} P_{1} \backslash a
\end{equation}

Two choices were considered for the problem of building a parser in TMACS. The first choice was to build
a new parser, which required much more resources and the second choice was to define a grammar
and to use a parser generator (compiler-compiler, compiler generator) software to generate the parser source code.
In formal language theory, a context-free grammar \cite{Chomsky} is a grammar in which every production 
rule has the form 
$V \rightarrow w$,
where V e is a nonterminal symbol, and w is a string of terminal and/or nonterminal symbols 
(w can be empty). Obviously, the defined BNF grammar for describing the CCS process is a CFG. 
Deterministic context-free grammars \cite{Chomsky} are grammars that can be recognized by a 
deterministic pushdown automaton or equivalently, grammars that can be recognized by a LR (left to right) 
parser \cite{Compilers}. Deterministic context-free grammars are a proper subset of the context-free grammar 
family \cite{Chomsky}. %Although, the defined grammar is a non-deterministic context-free grammar, modern parsers have a look ahead feature, which means that the parser will not make a parsing decision until it reads ahead several tokens. This feature alows us to use a parser generator software, that will generate LR or LL parser which is able to recognize the non-deterministic grammar.

\subsection{Generating the parser}
ANTLR (ANother Tool for Language Recognition) \cite{ANTLR} is a parser generator that was used 
to generate the parser for the CCS grammar in TMACS. ANTLR uses LL(*) parsing and also allows generating 
parsers, lexers, tree parsers and combined lexer parsers. It also automatically generates 
abstract syntax trees \cite{Compilers}, by providing operators and rewrite rules to guide the tree construction,
which can be further processed with a tree parser, to build an abstract model of the AST using some programming language.
%A language is specified by using a context-free grammar which is expressed using Extended Backus Naur Form (EBNF) \cite{Compilers}. In short, an LL parser is a top-down parser which parses the input from left to right and constructs a leftmost derivation of the sentence. An LL parser is called an LL(*) parser if it is not restricted to a finite k tokens of lookahead, but can make parsing decisions by recognizing whether the following tokens belong to a regular language. This is achieved by building temporary deterministic finite automaton, which are maintained and used until the parser makes a decision. This feature and some more optimization for the LL parser were published in recent years which made this kind of parser generators popular and favorable \cite{Compilers}.

ANTLR was used to generate lexer, parser and a well defined abstract syntax tree which represents a tree representation 
of the abstract syntactic structure of the parsed sentence. The term abstract is used in a sense 
that the tree does not represent every single detail that exists in the language syntax, 
e.g., parentheses are implicit in the tree structure. From a development point of view it is 
much easier to work with trees than with a list of recognized tokens. One example of an abstract syntax tree
is shown in Fig.~\ref{fig:ast_example} which is the result of parsing the expression: 

\begin{equation}\label{eq:ccs_example}
 A=( b.B \hspace{1 mm} | \hspace{1 mm} c.D \hspace{1 mm} + \hspace{1 mm} d.D )
\backslash \left\lbrace b, \hspace{1 mm} c \right\rbrace 
\end{equation}

\begin{figure}[h]
\centering
\includegraphics[width=3.0in]{ast_example}
\caption{Example of an abstract syntax tree}
\label{fig:ast_example}
\end{figure}

\subsection{CCS domain model and labeled graph generation}
Although working directly with abstract syntax trees and performing all algorithms on them is possible, it causes 
a limitation in future changes, where even a small change in the grammar and/or in the structure of the
generated abstract syntax trees causes a change in the implemented algorithms. Because of this,
a specific domain model was built along with a domain builder algorithm, which has corresponding
abstractions for all CCS operators, processes and actions. The input of the domain builder algorithm 
is an abstract syntax tree, and the output is a fully built domain model. 
%The algorithms for constructing a labeled graph were implemented on the domain model because it is not expected for the domain model structure to change much in the future. The domain model is also a tree-like structure, so it is as easy to work with, as with the abstract syntax tree. 

The algorithm for generation of labeled transition system implemented in TMACS is a recursive algorithm which 
traverses the tree structure of objects in the domain model and performs SOS rule every time it reaches an operation.
In this fashion all SOS transformation are performed on the domain and as result a new graph
structure is created which represents the labeled transition system which can be easily exported to a file
in Aldebaran format.

%\subsection{Workflow of operations}
%In Fig.~\ref{fig:workflow} the workflow of all operations that are executed for constructing a labeled transition system graph from a 
%CCS expression is shown. Every operation is done as a standalone algorithm independent from the other operations,
%that has input and output shown in the figure. The modular design was deliberately chosen in order to help achieve better 
%testing and maintenance of the source code. 


%\begin{figure}[h]
%\centering
%\includegraphics[height=3.0in]{workflow}
%\caption{Workflow of all operations for producing a labeled transition system graph from a CCS expression}
%\label{fig:workflow}
%\end{figure}
