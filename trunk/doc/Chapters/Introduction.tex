\section{Introduction}

Due to the increasing complexity of the real-life systems, manual analysis is very hard and often impossible. Therefore, having computer support for modeling, analysing, and verifying system behaviour is of an essential importance. This implies that the systems' properties and behaviour need to be described by formal methods with an explicit formal semantics.

One of the process languages used to formally describe and analyse any collection of interacting processes is the Calculus of Communication Systems (CCS), process algebra introduced by Robin Milner and based on the message-passing paradigm \cite{Milner1}\cite{Milner2}. CCS, like any process algebra, can be exploited to describe both specifications of the expected behaviours of the processes and their implementations \cite{HandbookProcessAlgebra}. The standard semantic model for CCS and various other process languages is the notion of labeled transition systems, a model first used by Keller to study concurrent systems \cite{Keller}. In order to analyze process behaviour and establish formally whether two processes offer the same behavior, different notions of behavioral equivalences over (states of) labeled transition systems have been proposed \cite{ReactiveSystems}.

One of the key behavioral equivalences is the strong bisimilarity \cite{Park} which relates processes whose behaviours are indistinguishable from each other \cite{UnderstandingConcurrentSystems}. Weak bisimilarity \cite{Milner1}\cite{Milner3} is a looser equivalence that abstracts away from internal (non-observable) actions. These and other behavioral equivalences can be employed to reduce the size of the state space of a labeled transition system and to check the equivalence between the behaviours of two labeled transition systems. This finds an extensive application in model checking and formal verification\cite{ModelChecking}\cite{ReactiveSystems}.

The notion of Hennessy-Milner logic \cite{HennessyMilner} is used to characterize the behavioural properties of systems modeled semantically as labeled transition systems. As a more powerful language for specifying properties of processes, the extension of the Hennessy-Milner logic with recursive definitions of formulas \cite{Larsen} is often employed. It entails the 'strong until' and 'weak until' operators to allow for recursive definitions \cite{ReactiveSystems}.

This paper presents TMACS, a tool that can be used to automate the process of modeling, manipulation and analysis of concurrent systems described by the CCS process language. TMACS stands for Tool for Modeling, manipulation, and Analysis of Concurrent Systems. It is given as a Java executable jar library [REF] with very simple Graphical User Interface (GUI). The tool is functional enough to be able to satisfy the basic requirements for modeling, specification, and verification. TMACS can parse CCS expressions, generate labeled transition system from CCS in Aldebaran format [REF], reduce a labeled transition system to its canonical form with respect to strong or weak bisimilarity and check whether two labeled transition systems are strongly or weakly bisimilar. Additionally, it can parse Hennessy-Milner formulas and implements the Hennessy-Milner logic operators for recursive definitions.

\subsection{Related work} 

Different tools for modeling, specification and verification of concurrent reactive systems have been developed over the past two decades. Probably the most famous and most commonly used ones, especially in the academic environment, are the Edinburgh Concurrency Workbench (CWB) \cite{CWB} and micro Common Representation Language 2 (mCRL2) \cite{mCRL2}. 

The Edinburgh Concurrency Workbench (CWB) is a tool for analysis of cuncurrent systems, which allows for equivalence, preorder and model checking using a variety of different process semantics. It also allows defining of behaviours in an extended version of CCS and perform various analysis of these behaviours, such as checking various semantic equivalences, e.g. strong and weak bisimilarity check. We used CWB to validate our tool, e.g. for checking CCS validity and also for testing algorithms for strong and weak bisimularity of labeled graphs. Although CWB covers much of the functionality of our tool and more, CWB has a command interpreter interface that is more difficult to work with, unlike our tool that has a Graphical User Interface (GUI), and is very intuitive. As far as we know the CWB does not have the functionality for exporting LTS graphs in Aldebaran format, that our tool has. 

The micro Common Representation Language 2 (mCRL2), successor of $\mu$CRL, is a formal specification language that can be used to specify and analyse the behaviour of distrubuted systems and protocols. Its accompanying toolset contains extensive collection of tools to automatically translate any mCRL2 specification to a linear process, manipulate and simulate linear processes, generate the state space associated with a linear process, manipulate and visualize state spaces, etc. All mCRL2 tools can be used from the command line, but mCRL2 has an enhanced Graphical User Interface (GUI) as well which makes it very user-friendly and easy to use. We used mCRL2 also to validate and test our tool, e.g. for testing the algorithms for minimization and comparison of labeled graphs using strong and weak bisimularity. However, to our knowledge, mCRL2 does not provide the possibility to define systems' behaviour in the CCS process language nor to specify systems' properties using HML logic. Also, even though mCRL2 supports minimization modulo strong and weak bisimulation equivalence, it does not output the computed bisimulation, feature that we have implemented in our tool.  

In addition it is worth mentioning some other tools as well: the Mobiility Workbench (MWB) \cite{MWB}, tool for manipulating and analyzing mobile concurrent systems, Construction and Analysis of Distributed Processes (CDAP) \cite{CDAP}, toolbox for the design of communication protocols and distributed systems, etc...
 
\subsection{Outline} The contributions of this paper are organized as follows. In Section~\ref{sec:preliminaries} we introduce some of the basic terminology used throughout the report. Section~\ref{sec:parsing} is devoted to the implementation of the CCS process language and generation of labeled transition systems as a semantic model of process expressions, and it also discusses some of the choices made during the implementation. Section~\ref{sec:bisimulation} describes the process of reducing the size of the state space of a labeled transition system as well as checking the equivalence between two labeled transition systems with respect to behavioural equivalences such as strong bisimilarity and observational equivalence (weak bisimilarity). It includes implementation details of two algorithms for computing strong bisimulation equivalence, the naive algorithm \cite{ReactiveSystems} and the advanced algorithm due to Fernandez \cite{Fernandez}. It also entails a description of the saturation technique together with a respective algorithm for saturating a labeled transition system which reduces the problem of computing weak bisimulation equivalence to computing strong bisimilarity over the saturated systems \cite{ReactiveSystems}. Section~\ref{sec:hml} explains how the implementation of Hennesy-Milner Logic (HML) works and gives implementation details for the 'strong until' and 'weak until' operators. Next, in Section~\ref{sec:application}, we illustrate the application of TMACS for modeling, specification and verification of two classical examples in the concurrency theory: the Alternating Bit Protocol and Peterson's mutual exclusion algorithm [REF]. Finally, we give some conclusions and some directions for future development of the tool in Section~\ref{sec:conclusion}. We also include four appendices. Appendix \cite{appendixA} contains the experimental results we got with analysis of the running times of our implementations of the two algorithms for computing bisimulation equivalence. Appendix \cite{appendixB} shows the syntax diagram for the grammar that recognizes CCS expressions, while Appendix \cite{appendixC} presents the grammar for Hennessy-Milner logic expressions. Appendix \cite{appendixD} includes some visual illustration of the tool's usage via screenshots of the graphical application.
