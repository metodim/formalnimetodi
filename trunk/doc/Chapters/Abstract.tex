\begin{abstract}
%\boldmath
The Calculus of Communicating Systems (CCS) is a young brunch in process calculus introduced 
by Robin Milner. CCS's theory has promising results in the field of modeling and analysing 
of reactive systems. Despite CCS's success, there are only few computer tools that are available
for modeling and analysis of CCS's theory. This paper is a technical report of an effort to
build such a tool. The tool is be able to parse CCS's expressions, it can build LTSs 
(labelled transition systems) from expressions and it can calculate if two LTS are bisimular
by using naive method or by using Fernandez's method. This tool if enough to perform modeling,
specification and verification of certain system.

\end{abstract}

% IEEEtran.cls defaults to using nonbold math in the Abstract.
% This preserves the distinction between vectors and scalars. However,
% if the conference you are submitting to favors bold math in the abstract,
% then you can use LaTeX's standard command \boldmath at the very start
% of the abstract to achieve this. Many IEEE journals/conferences frown on
% math in the abstract anyway.

% no keywords




% For peer review papers, you can put extra information on the cover
% page as needed:
% \ifCLASSOPTIONpeerreview
% \begin{center} \bfseries EDICS Category: 3-BBND \end{center}
% \fi
%
% For peerreview papers, this IEEEtran command inserts a page break and
% creates the second title. It will be ignored for other modes.